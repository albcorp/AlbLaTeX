%%%
%%% AlbAlgorithms/alb-avm-documentation.tex
%%%
%%%     See copyright notice and license in text.
%%%
%%%   - Documentation for the 'alb-avm' LaTeX package.
%%%
%%% $Id$
%%%



\documentclass[11pt,a4paper,oneside,titlepage]{alb-corp}



%
% URL Typesetting
%
% See: 'url.sty'.

\usepackage{url}


%
% The package being documented.

\usepackage{alb-avm}


%
% Sloppy Line Breaks
%
% Turn off careful line breaks and hyphenation.

\sloppy



\begin{document}



%%%
%%% TITLE
%%%

\albTitle{%
  Typesetting Attribute-Value Matrices Under \LaTeX{}%
}

\begin{albTitlePage}

  \albTitlePageSection{Author}

  Andrew Lincoln Burrow\\
  \texttt{albcorp@internode.on.net}


  \albTitlePageSection{Abstract}

  The \texttt{AlbAVM} package provides a single \texttt{alb-avm}
  \LaTeX{} package to typeset typed feature structures and inequated
  typed feature structures in attribute-value matrix (AVM) notation.
  This support is provided by an environment and associated markup
  commands.  In marking up AVM material, the resulting \LaTeX{}
  structure is isomorphic to the AVM structure.  The markup is placed in
  the \albLogo{} namespace.  The package is supported by an emacs lisp
  file customising \AUCTeX{} to automate the insertion of typed feature
  structures by a depth-first traversal of the AVM structure.


  \albTitlePageSection{Copyright}

  Copyright \copyright{} 1999--2006 Andrew Lincoln Burrow.\\
  This program may be distributed and/or modified under the conditions
  of the \LaTeX{} Project Public License, either version 1.3 of this
  license or (at your option) any later version.

  \medskip{}

  The latest version of this license is in
  \begin{quote}
    \url{http://www.latex-project.org/lppl.txt}
  \end{quote}
  and version 1.3 or later is part of all distributions of LaTeX version
  2005/12/01 or later.

  \medskip{}

  This work has the LPPL maintenance status `author-maintained'.

  \medskip{}

  This work consists of the files
  \begin{quote}
    \begin{flushleft}
      \url{alb-avm.sty} and
      \url{alb-avm-documentation.tex}.
    \end{flushleft}
  \end{quote}


  \albTitlePageSection{Version Information}

  \verb$Revision$\\
  \verb$Date$

\end{albTitlePage}



%%%
%%% INTRODUCTION
%%%

\section{Introduction}
\label{sec:alb-avm-documentation:intr}

The \texttt{alb-avm} \LaTeX{} package is designed to typeset typed
feature structures and inequated typed feature structures in
attribute-value matrix (AVM) notation.  The package is supported by an
emacs lisp file customising \AUCTeX{} to automate the insertion of typed
feature structures.

Section~\ref{sec:alb-avm-documentation:using-envir-comm} covers the use
of the \texttt{albAvm} environment and supporting commands.  The
examples also demonstrate a source code layout that is known to work
well with \AUCTeX{}.
Section~\ref{sec:alb-avm-documentation:auctex-cust} describes how the
commands are quickly entered under \AUCTeX{}.  This is the best way to
enter the AVM commands since it handles tags and source indentation
automatically.



%%%
%%% USING THE ENVIRONMENTS AND COMMANDS
%%%

\section{Using the Environments and Commands}
\label{sec:alb-avm-documentation:using-envir-comm}

Each topmost feature structure is represented by an instance of the
\texttt{albAvm} environment.  Within such an environment the
\texttt{albAvmType} command generates type labels, and the
\texttt{albAvmFeat} command generates feature-value pairs.  You must not
nest \texttt{albAvm} environments as nesting and tag numbering is
automated within the \texttt{albAvm} environment.  Instead, the
\texttt{albAvmFeat} and \texttt{albAvmTag} commands each take a second
arguments which is rendered as a nested substructure.



\subsection{Simple Feature Structures}
\label{sec:alb-avm-documentation:simple-feat-struct}

The most compact typed feature structure is a simple type.  It is the
only form expressed without nesting.  This simple form is just an
\texttt{albAvmType} command contained within an \texttt{albAvm}
environment, viz:
\begin{quote}
\begin{verbatim}
\begin{albAvm}
  \albAvmType{universal}
\end{albAvm}
\end{verbatim}
\end{quote}
which yields
\begin{quote}
  \begin{albAvm}
    \albAvmType{universal}
  \end{albAvm}
\end{quote}

Feature-values requires nesting.  This is achieved by placing the
contents of the feature-value as the second argument to the
\texttt{albAvmFeat} command.  Note that an \texttt{albAvm}-like
environment is automatically wrapped around the second argument to
\texttt{albAvmFeat}.  For example,
\begin{quote}
\begin{verbatim}
\begin{albAvm}
  \albAvmType{child}
  \albAvmFeat{FATHER}{%
    \albAvmType{human}
    }
  \albAvmFeat{MOTHER}{%
    \albAvmType{human}
    }
\end{albAvm}
\end{verbatim}
\end{quote}
yields
\begin{quote}
  \begin{albAvm}
    \albAvmType{child}
    \albAvmFeat{FATHER}{%
      \albAvmType{human}
      }
    \albAvmFeat{MOTHER}{%
      \albAvmType{human}
      }
  \end{albAvm}
\end{quote}
Also, deeper nesting proceeds in the obvious way, so that
\begin{quote}
\begin{verbatim}
\begin{albAvm}
  \albAvmType{child}
  \albAvmFeat{FATHER}{%
    \albAvmType{human}
    \albAvmFeat{MOTHER}{%
      \albAvmType{human}
      }
    \albAvmFeat{NAME}{%
      \albAvmType{name}
      }
    }
  \albAvmFeat{MOTHER}{%
    \albAvmType{human}
    }
\end{albAvm}
\end{verbatim}
\end{quote}
yields
\begin{quote}
  \begin{albAvm}
    \albAvmType{child}
    \albAvmFeat{FATHER}{%
      \albAvmType{human}
      \albAvmFeat{MOTHER}{%
        \albAvmType{human}
        }
      \albAvmFeat{NAME}{%
        \albAvmType{name}
        }
      }
    \albAvmFeat{MOTHER}{%
      \albAvmType{human}
      }
  \end{albAvm}
\end{quote}


\subsection{Feature Structures with Structure Sharing}
\label{sec:alb-avm-documentation:feat-struct-with-struct-s}

The \texttt{alb-avm} commands also provide for substructure tagging.
This is not automated by \LaTeX{} counters: automatic tags implemented
by extending the \LaTeX{} mechanism for labels and cross references is
beyond the scope of this implementation.  Instead tags must be
explicitly entered.  For a substructure to be tagged, place the tags as
the first argument to the \texttt{albAvmTag} command and place the
contents of the substructure as the second argument.  The
\texttt{albAvmTag} command occurs in place of the substructure.
Likewise, to refer to a substructure place the substructure's tag as the
only argument to the \texttt{albAvmRef} command.  This command also
occurs in place of the substructure.  For example, the liar sentence is
entered as
\begin{quote}
\begin{verbatim}
\begin{albAvm}
  \albAvmTag{1}{%
    \albAvmType{false}
    \albAvmFeat{ARG}{%
      \albAvmRef{1}
      }
    }
\end{albAvm}
\end{verbatim}
\end{quote}
which yields
\begin{quote}
  \begin{albAvm}
    \albAvmTag{1}{%
      \albAvmType{false}
      \albAvmFeat{ARG}{%
        \albAvmRef{1}
        }
      }
  \end{albAvm}
\end{quote}


\subsection{Inequated Feature Structures}
\label{sec:alb-avm-documentation:ineq-feat-struct}

Finally, provision is included for inequated feature structures.
Inequated feature structures are also entered using the \texttt{albAvm}
environment.  The initial part of the environment's body is entered like
a feature structure without inequations, excepting that all
substructures referred to in the inequations must be tagged.  The
remainder of the environment contains a comma separated list of
inequations.  Each inequation is entered with the \texttt{albAvmIneqtn}
command.  This command takes two arguments, which are simply the tags of
the substructures to be inequated.  For example,
\begin{quote}
\begin{verbatim}
\begin{albAvm}
  \albAvmType{house}
  \albAvmFeat{BEDROOM}{%
    \albAvmTag{1}{%
      \albAvmType{room}
      }
    }
  \albAvmFeat{KITCHEN}{%
    \albAvmTag{2}{%
      \albAvmType{room}
      }
    }
  \albAvmFeat{DININGROOM}{%
    \albAvmTag{3}{%
      \albAvmType{room}
      }
    }
  \albAvmFeat{LIVINGROOM}{%
    \albAvmTag{4}{%
      \albAvmType{room}
      }
    }
  \albAvmFeat{BATHROOM}{%
    \albAvmTag{5}{%
      \albAvmType{room}
      }
    }
  \albAvmIneqtn{1}{2}, \albAvmIneqtn{1}{5},
  \albAvmIneqtn{2}{5}, \albAvmIneqtn{3}{5},
  \albAvmIneqtn{4}{5}
\end{albAvmIneq}
\end{verbatim}
\end{quote}
yields
\begin{quote}
  \begin{albAvm}
    \albAvmType{house}
    \albAvmFeat{BEDROOM}{%
      \albAvmTag{1}{%
        \albAvmType{room}
        }
      }
    \albAvmFeat{KITCHEN}{%
      \albAvmTag{2}{%
        \albAvmType{room}
        }
      }
    \albAvmFeat{DININGROOM}{%
      \albAvmTag{3}{%
        \albAvmType{room}
        }
      }
    \albAvmFeat{LIVINGROOM}{%
      \albAvmTag{4}{%
        \albAvmType{room}
        }
      }
    \albAvmFeat{BATHROOM}{%
      \albAvmTag{5}{%
        \albAvmType{room}
        }
      }
    \albAvmIneqtn{1}{2}, \albAvmIneqtn{1}{5},
    \albAvmIneqtn{2}{5}, \albAvmIneqtn{3}{5},
    \albAvmIneqtn{4}{5}
  \end{albAvm}
\end{quote}



%%%
%%% AUCTEX CUSTOMISATIONS
%%%

\section{\AUCTeX{} Customisations}
\label{sec:alb-avm-documentation:auctex-cust}

Under \AUCTeX{} the file \texttt{alb-avm.el} is automatically loaded
(subject to certain \AUCTeX{} configuration options).  This customises
\AUCTeX{} to automate the entry of AVM notation through a collection of
mutually recursive functions that interrogate the user for AVM input.

Automatic entry of AVM notation is triggered when an \texttt{albAvm}
environment is inserted by the \texttt{LaTeX-environment} command.  By
default \AUCTeX{} binds \texttt{LaTeX-environment} to the keys
\texttt{C-c C-e}.  Automatic entry corresponds to a depth-first walk of
the feature structure.  At each substructure the user is prompted for a
description of the substructure.  The prompt contains the current
feature path to the substructure.

The first question about a substructure concerns its tag.  The history
list contains the tags entered for the feature structure.  If no value
is entered, then the tag is omitted; if the tag matches an existing tag,
then the substructure is represented by a simple reference to the
existing substructure; otherwise, the substructure is tagged with the
value.

Given a substructure has been allocated a new tag or no tag at all, it
must be explicitly constructed.  In this case the second question about
a substructure concerns its type.  The history list contains the types
entered during the current editing session.  If no value is entered,
then the substructure defaults to \texttt{universal} with the no
feature-values; otherwise, the type is recorded and feature-values
collected.  Feature-values are collected while non-empty strings are
returned for the features.  The history list contains the features
entered during the current editing session.  Recursion occurs in the
entry of the value part of the feature-values because each feature-value
is itself a substructure.

To prevent confusion while recurring into the substructures the prompt
strings are prefixed by the current path.  In addition, the source code
is incrementally entered with balanced parenthesis and indentation so
that it offers a useful cue to the substructure being defined.

As an exercise attempt to enter this feature structure:

\begin{quote}
  \begin{albAvm}
    \albAvmType{child}
    \albAvmFeat{FATHER}{%
      \albAvmType{human}
      \albAvmFeat{MOTHER}{%
        \albAvmType{human}
        }
      \albAvmFeat{NAME}{%
        \albAvmType{name}
        }
      }
    \albAvmFeat{MOTHER}{%
      \albAvmType{human}
      }
  \end{albAvm}
\end{quote}



\end{document}



%%% Local Variables:
%%% mode: latex
%%% TeX-master: t
%%% End:
