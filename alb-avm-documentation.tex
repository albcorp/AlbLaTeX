%%% 
%%% Documentation/avm-examples.tex
%%% 
%%%   - Examples for the _alb-avm_ package.
%%% 
%%% $Id$
%%% 



\documentclass[a4paper,11pt]{article}



% 
% AVM notation for typed feature structures.

\usepackage{alb-avm}


% 
% Typeset version of AUCTeX name.

\newcommand{\AUCTeX}[0]{AUC\TeX{}}



\begin{document}



%%% 
%%% TITLE
%%% 

\title{%
  Attribute-Value Matrix Depiction of Typed Feature Structures%
  }

\author{%
  Andrew Burrow, \texttt{alburrow@cs.adelaide.edu.au}\\
  Department of Computer Science,\\
  University of Adelaide,\\
  Adelaide SA 5005\\
  Australia%
  }

\date{$Revision$, $Date$}

\maketitle{}



%%% 
%%% INTRODUCTION
%%% 

\section{Introduction}
\label{sec:introduction}

The \texttt{alb-avm} package provides a single environment for type
setting typed feature structures and inequated typed feature structures
in attribute-value matrix (AVM) notation.  The package is supported by
an emacs lisp file customising \AUCTeX{} to automate the insertion of
typed feature structures.

Section~\ref{sec:using-texttt-envir} covers the use of the
\texttt{albAvm} environment and supporting macros.  The examples also
demonstrate a source code layout that is known to work well with
\AUCTeX{}.  Section~\ref{sec:auct-cust} describes how the commands are
quickly entered under \AUCTeX{}.  This is the best way to enter the AVM
macros since it handles tags and source indentation automatically.



%%% 
%%% USING THE \texttt{albAvm} ENVIRONMENT AND MACROS
%%% 

\section{Using the \texttt{albAvm} Environment and Macros}
\label{sec:using-texttt-envir}

Each topmost feature structure is represented by an instance of the
\texttt{albAvm} environment.  Within such an environment the
\texttt{albAvmType} command generates type labels, and the
\texttt{albAvmFeat} command generates feature-value pairs.  You must not
nest \texttt{albAvm} environments as nesting and tag numbering is
automated within the \texttt{albAvm} environment.  Instead, the
\texttt{albAvmFeat} and \texttt{albAvmTag} commands each take a second
arguments which is rendered as a nested substructure.


\subsection{Simple Feature Structures}
\label{sec:simple-feat-struct}

The most compact typed feature structure is a simple type.  It is the
only form expressed without nesting.  This simple form is just an
\texttt{albAvmType} command contained within an \texttt{albAvm}
environment, viz:
\begin{quote}
\begin{verbatim}
\begin{albAvm}
  \albAvmType{universal}
\end{albAvm}
\end{verbatim}
\end{quote}
which yields
\begin{quote}
  \begin{albAvm}
    \albAvmType{universal}
  \end{albAvm}
\end{quote}

Feature-values requires nesting.  This is achieved by placing the
contents of the feature-value as the second argument to the
\texttt{albAvmFeat} command.  Note that an \texttt{albAvm}-like
environment is automatically wrapped around the second argument to
\texttt{albAvmFeat}.  For example,
\begin{quote}
\begin{verbatim}
\begin{albAvm}
  \albAvmType{child}
  \albAvmFeat{FATHER}{%
    \albAvmType{human}
    }
  \albAvmFeat{MOTHER}{%
    \albAvmType{human}
    }
\end{albAvm}
\end{verbatim}
\end{quote}
yields
\begin{quote}  
  \begin{albAvm}
    \albAvmType{child}
    \albAvmFeat{FATHER}{%
      \albAvmType{human}
      }
    \albAvmFeat{MOTHER}{%
      \albAvmType{human}
      }
  \end{albAvm}
\end{quote}
Also, deeper nesting proceeds in the obvious way, so that 
\begin{quote}
\begin{verbatim}
\begin{albAvm}
  \albAvmType{child}
  \albAvmFeat{FATHER}{%
    \albAvmType{human}
    \albAvmFeat{MOTHER}{%
      \albAvmType{human}
      }
    \albAvmFeat{NAME}{%
      \albAvmType{name}
      }
    }
  \albAvmFeat{MOTHER}{%
    \albAvmType{human}
    }
\end{albAvm}
\end{verbatim}
\end{quote}
yields
\begin{quote}
  \begin{albAvm}
    \albAvmType{child}
    \albAvmFeat{FATHER}{%
      \albAvmType{human}
      \albAvmFeat{MOTHER}{%
        \albAvmType{human}
        }
      \albAvmFeat{NAME}{%
        \albAvmType{name}
        }
      }
    \albAvmFeat{MOTHER}{%
      \albAvmType{human}
      }
  \end{albAvm}
\end{quote}


\subsection{Feature Structures with Structure Sharing}
\label{sec:feat-struct-with}

The \texttt{alb-avm} macros also provide for substructure tagging.  This
is not automated by \LaTeX{} counters: automatic tags implemented by
extending the \LaTeX{} mechanism for labels and cross references is
beyond the scope of this implementation.  Instead tags must be
explicitly entered.  For a substructure to be tagged, place the tags as
the first argument to the \texttt{albAvmTag} command and place the
contents of the substructure as the second argument.  The
\texttt{albAvmTag} command occurs in place of the substructure.
Likewise, to refer to a substructure place the substructure's tag as the
only argument to the \texttt{albAvmRef} command.  This command also
occurs in place of the substructure.  For example, the liar sentence is
entered as
\begin{quote}
\begin{verbatim}
\begin{albAvm}
  \albAvmTag{1}{%
    \albAvmType{false}
    \albAvmFeat{ARG}{%
      \albAvmRef{1}
      }
    }
\end{albAvm}
\end{verbatim}
\end{quote}
which yields
\begin{quote}
  \begin{albAvm}
    \albAvmTag{1}{%
      \albAvmType{false}
      \albAvmFeat{ARG}{%
        \albAvmRef{1}
        }
      }
  \end{albAvm}
\end{quote}


\subsection{Inequated Feature Structures}
\label{sec:ineq-feat-struct}

Finally, provision is included for inequated feature structures.
Inequated feature structures are also entered using the \texttt{albAvm}
environment.  The initial part of the environment's body is entered like
a feature structure without inequations, excepting that all
substructures referred to in the inequations must be tagged.  The
remainder of the environment contains a comma separated list of
inequations.  Each inequation is entered with the \texttt{albAvmIneqtn}
command.  This command takes two arguments, which are simply the tags of
the substructures to be inequated.  For example,
\begin{quote}
\begin{verbatim}
\begin{albAvm}
  \albAvmType{house}
  \albAvmFeat{BEDROOM}{%
    \albAvmTag{1}{%
      \albAvmType{room}
      }
    }
  \albAvmFeat{KITCHEN}{%
    \albAvmTag{2}{%
      \albAvmType{room}
      }
    }
  \albAvmFeat{DININGROOM}{%
    \albAvmTag{3}{%
      \albAvmType{room}
      }
    }
  \albAvmFeat{LIVINGROOM}{%
    \albAvmTag{4}{%
      \albAvmType{room}
      }
    }
  \albAvmFeat{BATHROOM}{%
    \albAvmTag{5}{%
      \albAvmType{room}
      }
    }
  \albAvmIneqtn{1}{2}, \albAvmIneqtn{1}{5},
  \albAvmIneqtn{2}{5}, \albAvmIneqtn{3}{5},
  \albAvmIneqtn{4}{5}
\end{albAvmIneq}
\end{verbatim}
\end{quote}
yields
\begin{quote}
  \begin{albAvm}
    \albAvmType{house}
    \albAvmFeat{BEDROOM}{%
      \albAvmTag{1}{%
        \albAvmType{room}
        }
      }
    \albAvmFeat{KITCHEN}{%
      \albAvmTag{2}{%
        \albAvmType{room}
        }
      }
    \albAvmFeat{DININGROOM}{%
      \albAvmTag{3}{%
        \albAvmType{room}
        }
      }
    \albAvmFeat{LIVINGROOM}{%
      \albAvmTag{4}{%
        \albAvmType{room}
        }
      }
    \albAvmFeat{BATHROOM}{%
      \albAvmTag{5}{%
        \albAvmType{room}
        }
      }
    \albAvmIneqtn{1}{2}, \albAvmIneqtn{1}{5},
    \albAvmIneqtn{2}{5}, \albAvmIneqtn{3}{5},
    \albAvmIneqtn{4}{5} 
  \end{albAvm}
\end{quote}
 


%%% 
%%% AUCTEX CUSTOMISATIONS
%%% 

\section{\AUCTeX{} Customisations}
\label{sec:auct-cust}

Under \AUCTeX{} the file \texttt{alb-avm.el} is automatically loaded
(subject to certain \AUCTeX{} configuration options).  This customises
\AUCTeX{} to automate the entry of AVM notation through a collection of
mutually recursive functions that interrogate the user for AVM input.

Automatic entry of AVM notation is triggered when an \texttt{albAvm}
environment is inserted by the \texttt{LaTeX-environment} command.  By
default \AUCTeX{} binds \texttt{LaTeX-environment} to the keys \texttt{C-c
  C-e}.  Automatic entry corresponds to a depth-first walk of the
feature structure.  At each substructure the user is prompted for
a description of the substructure.  The prompt contains the current
feature path to the substructure.

The first question about a substructure concerns its tag.  The history
list contains the tags entered for the feature structure.  If no value
is entered, then the tag is omitted; if the tag matches an existing tag,
then the substructure is represented by a simple reference to the
existing substructure; otherwise, the substructure is tagged with the
value.

Given a substructure has been allocated a new tag or no tag at all, it
must be explicitly constructed.  In this case the second question about
a substructure concerns its type.  The history list contains the types
entered during the current editing session.  If no value is entered,
then the substructure defaults to \texttt{universal} with the no
feature-values; otherwise, the type is recorded and feature-values
collected.  Feature-values are collected while non-empty strings are
returned for the features.  The history list contains the features
entered during the current editing session.  Recursion occurs in the
entry of the value part of the feature-values because each feature-value
is itself a substructure.

To prevent confusion while recursing into the substructures the prompt
strings are prefixed by the current path.  In addition, the source code
is incrementally entered with balanced parenthesis and indentation so
that it offers a useful cue to the substructure being defined.

As an exercise attempt to enter this feature structure:

\begin{quote}
  \begin{albAvm}
    \albAvmType{child}
    \albAvmFeat{FATHER}{%
      \albAvmType{human}
      \albAvmFeat{MOTHER}{%
        \albAvmType{human}
        }
      \albAvmFeat{NAME}{%
        \albAvmType{name}
        }
      }
    \albAvmFeat{MOTHER}{%
      \albAvmType{human}
      }
  \end{albAvm}
\end{quote}


\end{document}




%%% Local Variables: 
%%% mode: latex
%%% TeX-master: t
%%% End: 
