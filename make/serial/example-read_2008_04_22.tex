%
% :Precis: Example readings for 22 APR 2008
% :Contact: albcorp@gmail.com
% :Authors: Andrew Burrow
% :Copyright: 2005-2008, 2013 Andrew Lincoln Burrow
%


\chapter{22 APR 2008}
\label{cha:read_2008_04_22:22-apr-2008}

Papers collected on wiki systems.

%
%
% Background
%

\section{Background}
\label{sec:read_2008_04_22:backgr}

%
% volkel06:_semwik

\begin{albResearchNote}{%
    \citet{volkel06:_semwik}%
  }{%
    \glossary{wiki}, \glossary{REST}, \glossary{semantic~wiki}%
  }
\item Describes the \emph{SemWiki} wiki-engine and its RESTful
  architecture.
\item Demonstrates the separation of parser, user interface and data
  management components into lightweight, REST-style web services.
\end{albResearchNote}

%
% sauer07:_wikic

\begin{albResearchNote}{%
    \citet{sauer07:_wikic}%
  }{%
    \glossary{wiki}, \glossary{wiki-markup}%
  }
\item Describes the wiki markup language \emph{WikiCreole}, its
  development, and related work
\item Claims that wiki markup assists authors concentrate on content
  over formatting, and that wiki markup will continue to be in demand
\item Proposes that WikiCreole be a wiki markup system for use in all
  wiki systems as an alternate
\item Describes implementation modes that address the use of a mixture
  of wiki markup languages.  In particular, the markup is designed to
  prevent collisions with widely used markups
\end{albResearchNote}

%
% dekel07:_framew_study_use_wikis_knowl

\begin{albResearchNote}{%
    \citet{dekel07:_framew_study_use_wikis_knowl}%
  }{%
    \glossary{wiki}, \glossary{user~activity~tracking}%
  }
\item Describes a system for measuring the passive use of existing
  content in a wiki
\item Describes a client-side implementation that allows the collection
  of detailed information about the reading patterns of a user
\item Claims that, for a given developer, ``it may be possible to
  identify indirect evidence of things that could have been part of this
  context.''
\item Focuses on large scale data analysis over multiple sessions
\item Claims the distinguishing contribution is a framework for
  representing and accumulating the data
\item Reports a stream of viewport windows for a rendered page
\end{albResearchNote}

%
%
% Bibliography
%

\bibliographystyle{plainnat}
\bibliography{example-bibliography}

% Local Variables:
% mode: latex
% TeX-master: "example-journal"
% ispell-local-dictionary: "british"
% End:
