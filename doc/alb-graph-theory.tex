%%%
%%% AlbLaTeX/doc/alb-graph-theory.tex
%%%
%%%     See copyright notice and license in text.
%%%
%%%   - Documentation for the 'alb-graph-theory' LaTeX package.
%%%



\documentclass[11pt,a4paper,oneside,titlepage]{alb-corp}



%
% URL Typesetting
%
% See: 'url.sty'.

\usepackage{url}


%
% The package being documented.

\usepackage{alb-graph-theory}


%
% Sloppy Line Breaks
%
% Turn off careful line breaks and hyphenation.

\sloppy



\begin{document}



%%%
%%% TITLE
%%%

\albTitle{%
  Typesetting Graph Theory Under \LaTeX{}%
}

\begin{albTitlePage}

  \albTitlePageSection{Author}

  Andrew Lincoln Burrow\\
  \texttt{albcorp@gmail.com}


  \albTitlePageSection{Abstract}

  The \texttt{AlbGraphTheory} package provides a single
  \texttt{alb-graph-theory} \LaTeX{} package provide markup for graph
  theory.  It is designed to make the \LaTeX{} input more readable, to
  allow the actual symbols to be adjusted from a single file, to enable
  the typesetting of graph theoretic constructions to be designed in the
  context of the entire suite of required markup commands, and to place
  the extended markup in the \albLogo{} namespace.  The package is
  supported by an emacs lisp file customising \AUCTeX{}, which provides
  command name completion and argument prompting.


  \albTitlePageSection{Copyright}

  Copyright \copyright{} 2004--2006, 2013 Andrew Lincoln Burrow.\\
  This program may be distributed and/or modified under the conditions
  of the \LaTeX{} Project Public License, either version 1.3 of this
  license or (at your option) any later version.

  \medskip{}

  The latest version of this license is in
  \begin{quote}
    \url{http://www.latex-project.org/lppl.txt}
  \end{quote}
  and version 1.3 or later is part of all distributions of LaTeX version
  2005/12/01 or later.

  \medskip{}

  This work has the LPPL maintenance status `author-maintained'.

  \medskip{}

  This work consists of the files
  \begin{quote}
    \begin{flushleft}
      \url{alb-algorithms.sty}, \url{alb-avm.sty}, \url{alb-corp.cls},
      \url{alb-float-tools.sty}, \url{alb-graph-theory.sty},
      \url{alb-journal.cls}, \url{alb-order-theory.sty},
      \url{alb-proofs.sty}, \url{alb-theorems.sty},
      \url{alb-thesis.cls}, \url{alb-algorithms.tex}, \url{alb-avm.tex},
      \url{alb-corp.tex}, \url{alb-float-tools.tex},
      \url{alb-graph-theory.tex}, \url{alb-journal.tex},
      \url{alb-order-theory.tex}, \url{alb-proofs.tex},
      \url{alb-theorems.tex}, \url{alb-thesis.tex}.
      \url{alb-journal-glossary.ist}, \url{alb-journal-index.ist},
      \url{alb-thesis-glossary.ist}, and \url{alb-thesis-index.ist}.
    \end{flushleft}
  \end{quote}


  \albTitlePageSection{Version Information}

  \verb$Revision$\\
  \verb$Date$

\end{albTitlePage}



%%%
%%% INTRODUCTION
%%%

\section{Introduction}
\label{sec:alb-graph-theory-documentation:intr}

The \texttt{alb-graph-theory} \LaTeX{} package is designed to typeset
graph theory.  In practice, this means that it supplies a large
collection of simple math commands.  The main benefit is consistency,
and the ability to easily change a decision about typesetting for all
documents.



%%%
%%% USING THE COMMANDS AND ENVIRONMENTS
%%%

\section{Using the Commands and Environments}
\label{sec:alb-order-theory-documentation:using-comm-envir}

The commands and environments of the \texttt{alb-order-theory} \LaTeX{}
package are all used within math mode.  They accept arguments and
generate graph theory mathematical constructions.

\begin{quote}
  \begin{tabular}{p{0.5\textwidth}@{\qquad}p{0.3\textwidth}}
    %
    % The head accessor for an edge
    \verb$\albHead{e}$ & $\albHead{e}$ \\
    %
    % The tail accessor for an edge
    \verb$\albTail{e}$ & $\albTail{e}$ \\
    %
    % The two place function returning the distance
    \verb$\albDistance{u}{v}$ & $\albDistance{u}{v}$ \\
    %
    % The three place function returning the distance
    \verb$\albDistanceIn{G}{u}{v}$ & $\albDistanceIn{G}{u}{v}$
  \end{tabular}
\end{quote}



%%%
%%% AUCTEX CUSTOMISATIONS
%%%

\section{\AUCTeX\ Customisations}
\label{sec:alb-graph-theory-documentation:auctex-cust}

Under \AUCTeX{} the file \texttt{alb-graph-theory.el} is automatically
loaded whenever the \texttt{alb-graph-theory} package is used.  The
customisation adds the math commands to \AUCTeX{}.  This provides the
simple prompting for all the supplied math commands.

It may be worthwhile also loading the \texttt{alb-order-theory} package
because \texttt{alb-order-theory.el} causes a special equation number
counter to be stored as a local variable.
\texttt{alb-LaTeX-equation-counter} counts numbers assigned to labels
for equations.  This ensures dirty numbers are not reissued, as
reissuing a number could make stale references hard to detect.



\end{document}



%%% Local Variables:
%%% mode: latex
%%% TeX-master: t
%%% End:
