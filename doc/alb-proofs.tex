%%%
%%% AlbLaTeX/doc/alb-proofs.tex
%%%
%%%     See copyright notice and license in text.
%%%
%%%   - Documentation for the 'alb-proofs' LaTeX package.
%%%



\documentclass[11pt,a4paper,oneside,titlepage]{alb-corp}



%
% URL Typesetting
%
% See: 'url.sty'.

\usepackage{url}


%
% Author Year Citations
%
% See: 'natbib.dvi'.

\usepackage[sectionbib,round]{natbib}


%
% Set and order theoretic mathematical constructs.
%
% See: 'alb-order-theory-documentation.pdf'.

\usepackage{alb-order-theory}


%
% The package being documented.

\usepackage{alb-proofs}


%
% Sloppy Line Breaks
%
% Turn off careful line breaks and hyphenation.

\sloppy



\begin{document}


%%%
%%% TITLE
%%%

\albTitle{%
  Typesetting Lamport's Structured Proofs Under \LaTeX{}%
}


\begin{albTitlePage}

  \albTitlePageSection{Author}

  Andrew Lincoln Burrow\\
  \texttt{albcorp@gmail.com}


  \albTitlePageSection{Abstract}

  The \texttt{AlbProofs} package provides a single \texttt{alb-proofs}
  \LaTeX{} package to implement Leslie Lamport's structured proofs
  \citep{lamport93:_how_write_proof}.  While Lamport has provided his
  own package, this package is designed to rectify certain deficiencies.
  In particular, it provides improved type setting of labels for proof
  resources and proof steps, markup structure which is isomorphic to the
  proof's block structure, effective support for cross-referencing with
  \AUCTeX{} and \RefTeX{}, and markup in the \albLogo{} name space.  The
  package is supported by an emacs lisp file customising \AUCTeX{} and
  \RefTeX{} which constructs labels in the file namespace.


  \albTitlePageSection{Copyright}

  Copyright \copyright{} 2001--2006, 2013 Andrew Lincoln Burrow.\\
  This program may be distributed and/or modified under the conditions
  of the \LaTeX{} Project Public License, either version 1.3 of this
  license or (at your option) any later version.

  \medskip{}

  The latest version of this license is in
  \begin{quote}
    \url{http://www.latex-project.org/lppl.txt}
  \end{quote}
  and version 1.3 or later is part of all distributions of LaTeX version
  2005/12/01 or later.

  \medskip{}

  This work has the LPPL maintenance status `author-maintained'.

  \medskip{}

  This work consists of the files
  \begin{quote}
    \begin{flushleft}
      \url{alb-algorithms.sty}, \url{alb-avm.sty}, \url{alb-corp.cls},
      \url{alb-float-tools.sty}, \url{alb-graph-theory.sty},
      \url{alb-journal.cls}, \url{alb-order-theory.sty},
      \url{alb-proofs.sty}, \url{alb-theorems.sty},
      \url{alb-thesis.cls}, \url{alb-algorithms.tex}, \url{alb-avm.tex},
      \url{alb-corp.tex}, \url{alb-float-tools.tex},
      \url{alb-graph-theory.tex}, \url{alb-journal.tex},
      \url{alb-order-theory.tex}, \url{alb-proofs.tex},
      \url{alb-theorems.tex}, \url{alb-thesis.tex}.
      \url{alb-journal-glossary.ist}, \url{alb-journal-index.ist},
      \url{alb-thesis-glossary.ist}, and \url{alb-thesis-index.ist}.
    \end{flushleft}
  \end{quote}


  \albTitlePageSection{Version Information}

  \verb$Revision$\\
  \verb$Date$

\end{albTitlePage}



%%%
%%% INTRODUCTION
%%%

\section{Introduction}
\label{sec:alb-proofs-examples:intr}

The \texttt{alb-proofs} \LaTeX{} package implements Leslie Lamport's
structured proofs described in \citep{lamport93:_how_write_proof}.  A
structured proof decomposes each argument into a sequence of steps.
First come assumptions, then assertions, and finally a conclusion.  Each
assumption introduces a variable or condition, and each assertion is
itself the goal of a nested proof.  Together an assertion and its
subproof are termed a \emph{proof step}.

The purpose of the structured proof is to make all steps explicit and to
sequester detail into the nested subproofs.  A useful approach is to
first scan the top level assertions to take in the outline, and then to
examine the subproofs in detail.  Ideally, the proof justifies each
assertion by reference to a supporting mathematical proposition and a
list of accessible assumptions and assertions.  In practice, the proof
is resolved until each assertion can be argued succinctly in a few short
sentences.

Figure~\ref{fig:alb-proofs-documentation:example-syntax-struct-pro}
shows a possible layout for a small proof.  It demonstrates two levels
of nesting, where there are fresh assumptions in each nesting: the
outer, and the nesting under $\albTuple{1}$2.

It is important to understand the scope of assumptions and assertions.
Access is determined by a rule that recurs on the nesting in the proof.
Note that each assertion occurs in a sequence of steps with a common
parent.  The assertion enjoys the assumptions and assertions that
precede it in the list, and also those enjoyed by its parent.  For
example, in
Figure~\ref{fig:alb-proofs-documentation:example-syntax-struct-pro} the
proof step at $\albTuple{2}$1 can make use of assumptions
$\albTuple{1}$i, $\albTuple{1}$ii, and $\albTuple{2}$i and assertion
$\albTuple{1}$1; it cannot use $\albTuple{1}$2 which does not precede
it, but instead contains it; conversely, $\albTuple{1}$3 cannot use
$\albTuple{2}$i, $\albTuple{2}$1, and $\albTuple{2}$2 but can use
$\albTuple{1}$i, $\albTuple{1}$ii, $\albTuple{1}$1, and $\albTuple{1}$2
because they precede it in the only sequence in which it participates.
Note that access is determined by the structure rather than the
labelling.

The labelling of resources exploits the selectivity of the access rule.
Since a proof step only has access to resources in one sequence per
level of nesting, the sequence is uniquely identified by a nesting
level.  Therefore, rather than generate labels of the form 1.2.1.1,
labels concatenate the nesting level, written within angle brackets, and
a further counter.  Assumptions use Roman numerals while assertions use
Arabic numerals --- this is analogous to the page numbering in a book.

The headings on assertions are extended to clarify the use of
mathematical induction.  In this case, the top-level assertions are
marked ``\textsc{Assert Base}'' or ``\textsc{Assert Induction}'', and
the conclusion is marked ``\textsc{Q.E.D. by Mathematical Induction}''.
The top-level assertions are restatements of the goal narrowed for the
particular case.

\begin{figure}
  \begin{albProof}
    \begin{albAssumptions}
    \item \label{as:intr:1} %
      \albAssume{\emph{assumption}}

    \item \label{as:intr:2} %
      \albAssume{\emph{assumption}}
    \end{albAssumptions}

    \begin{albAssertions}
    \item \label{pf:intr:1} %
      \albAssert{\emph{assertion}}

      Proof of the assertion.

    \item \label{pf:intr:2} %
      \albAssert{\emph{assertion}}

      \begin{albAssumptions}
      \item \label{as:intr:3} %
        \albAssume{\emph{assumption}}
      \end{albAssumptions}

      \begin{albAssertions}
      \item \label{pf:intr:3} %
        \albAssert{\emph{assertion}}

        Proof of the assertion.

      \item %
        \albQED{}

        Explanation of the conclusion.
      \end{albAssertions}

    \item %
      \albQED{}

      Explanation of the conclusion.
    \end{albAssertions}
  \end{albProof}
  \caption[An example of the syntax of a structured proof]{%
    An example of the syntax of a structured proof.}
  \label{fig:alb-proofs-documentation:example-syntax-struct-pro}
\end{figure}



%%%
%%% USING THE COMMANDS AND ENVIRONMENTS
%%%

\section{Using the Commands and Environments}
\label{sec:alb-proofs-documentation:using-comm-envir}

The environments of \texttt{alb-proofs} place certain syntactic
restrictions on their use.  This section concerns these restrictions and
the typographic meaning of the markup.
Section~\ref{sec:alb-proofs-documentation:auctex-cust} describes how the
\AUCTeX{} customisation eases the burden of entering syntactically
correct \LaTeX{} code.



\subsection{Structured Proof Environment}
\label{sec:alb-proofs-documentation:struct-proof-envir}

The \texttt{albProof} environment replaces the AMS\LaTeX{}
\texttt{proof} environment to support structured proofs.  Content within
a structured proof is recursively composed, where each level comprises
an optional \texttt{albSketch} environment, an optional
\texttt{albAssumptions} environment, and a mandatory
\texttt{albAssertions} environment.  Recursion occurs because items in
\texttt{albAssertions} environments can expand into subproofs.

\begin{description}
\item[\albLtxEnv{albProof}\albLtxOpt{name}] Compose a proof for a
  mathematical proposition.  \albLtxPrm{name} is the name of the
  proposition being proved.  This is typically a reference to a
  theorem-like environment from \texttt{alb-theorems}.  The environment
  contains an optional \texttt{albSketch} environment, an optional
  \texttt{albAssumptions} environment, and a mandatory
  \texttt{albAssertions} environment.

\item[\albLtxEnv{albSketch}] Present a sketch of the proof.  The intent
  is to allow comments on the proof strategy to imrpove readability.

\item[\albLtxEnv{albAssumptions}] Enumerate the assumptions introduced
  at the current proof step.  Each assumption is introduced by an
  \texttt{item} command that may be labelled.  The assumption should be
  marked up using either \texttt{albAssume} or \texttt{albSuppose}.

\item[\albLtxCmd{albAssume}\albLtxArg{assumption}] Typeset an assumption
  in a proof.  \albLtxPrm{assumption} is typically a mathematical
  expression.  \texttt{albAssume} should be used inside
  \texttt{albAssumptions} to indicate assumptions unless the proof is by
  contradiction.  For example,
  \begin{quote}
\begin{verbatim}
\albAssume{$x \in \albNaturals$}
\end{verbatim}
  \end{quote}
  produces the following.
  \begin{quote}
    \albAssume{$x \in \albNaturals$}
  \end{quote}

\item[\albLtxCmd{albSuppose}\albLtxArg{supposition}] Typeset a
  supposition in a proof by contradiction.  \albLtxPrm{supposition} is
  typically a mathematical expression.  \texttt{albSuppose} should be
  used inside \texttt{albAssumptions}.  For example,
  \begin{quote}
\begin{verbatim}
\albSuppose{$x \in \albNaturals$}
\end{verbatim}
  \end{quote}
  produces the following.
  \begin{quote}
    \albSuppose{$x \in \albNaturals$}
  \end{quote}

\item[\albLtxEnv{albAssertions}] Enumerate the sequence of assertions
  establishing the truth of a proof step.  Each assertion is introduced
  by an \texttt{item} command that may be labelled.  The assertion
  should be marked up using one of the commands \texttt{albAssert},
  \texttt{albAssertBase}, \texttt{albAssertInduction}, \texttt{albQED},
  \texttt{albQEDbyContradiction}, or \texttt{albQEDbyInduction}.  These
  commands correspond to the various stages in a proof by cases,
  contradiction, or mathematical induction.

  The proof of the assertion follows as either a free form description
  or a nested proof detailed as a \texttt{albSketch},
  \texttt{albAssumptions}, and \texttt{albAssertions} sequence.

\item[\albLtxCmd{albAssert}\albLtxArg{assertion}] Typeset an assertion
  in a proof.  \albLtxPrm{assertion} is typically a mathematical
  expression.  \texttt{albAssert} should be used inside
  \texttt{albAssertions}.  For example,
  \begin{quote}
\begin{verbatim}
\albAssert{$F(x) \implies x \in \albNaturals$}
\end{verbatim}
  \end{quote}
  produces the following.
  \begin{quote}
    \albAssert{$F(x) \implies x \in \albNaturals$}
  \end{quote}

\item[\albLtxCmd{albAssertBase}\albLtxArg{assertion}] Typeset the base
  assertion in a proof by induction.  \albLtxPrm{assertion} is typically
  a mathematical expression.  \texttt{albAssertBase} should be used
  inside \texttt{albAssertions}.  For example,
  \begin{quote}
\begin{verbatim}
\albAssertBase{$F(0) \in \albNaturals$}
\end{verbatim}
  \end{quote}
  produces the following.
  \begin{quote}
    \albAssertBase{$F(0) \in \albNaturals$}
  \end{quote}

\item[\albLtxCmd{albAssertInduction}\albLtxArg{assertion}] Typeset the
  assertion of induction in a proof by induction.  \albLtxPrm{assertion}
  is typically a mathematical expression.  \texttt{albAssertInduction}
  should be used inside \texttt{albAssertions}.  For example,
  \begin{quote}
\begin{verbatim}
\albAssertInduction{$F(n) \in \albNaturals
  \implies F(n + 1) \in \albNaturals$}
\end{verbatim}
  \end{quote}
  produces the following.
  \begin{quote}
    \albAssertInduction{$F(n) \in \albNaturals \implies F(n + 1) \in
      \albNaturals$}
  \end{quote}

\item[\albLtxCmd{albQED}, \albLtxCmd{albQEDbyContradiction}, and
  \albLtxCmd{albQEDbyInduction}] Typeset the assertion that the sequence
  of proof steps is complete.  The commands each terminate a sequence of
  proof steps according to the form of proof.  For example,
  \begin{quote}
\begin{verbatim}
\albQED{}
\albQEDbyContradiction{}
\albQEDbyInduction{}
\end{verbatim}
  \end{quote}
  respectively produce the following.
  \begin{quote}
    \albQED{}
    \albQEDbyContradiction{}
    \albQEDbyInduction{}
  \end{quote}
\end{description}



%%%
%%% AUCTEX CUSTOMISATIONS
%%%

\section{\AUCTeX{} Customisations}
\label{sec:alb-proofs-documentation:auctex-cust}

Under \AUCTeX{} the file \texttt{alb-proofs.el} is automatically loaded
whenever the \texttt{alb-proofs} package is used.  The customisation
adds support for the commands and environments from \texttt{alb-proofs}.
In addition, the \texttt{reftex-label} command (\texttt{C-c (})
constructs labels for proof steps, and the \texttt{reftex-reference}
command (\texttt{C-c )}) formats the context lines to better indicate
the typeset output.



%%%
%%% BIBLIOGRAPHY
%%%

\bibliographystyle{plainnat}
\bibliography{alb-bibliography}



\end{document}



%%% Local Variables:
%%% mode: latex
%%% TeX-master: t
%%% End:
