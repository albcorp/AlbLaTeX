%%%
%%% AlbLaTeX/doc/alb-theorems.tex
%%%
%%%     See copyright notice and license in text.
%%%
%%%   - Documentation for 'alb-theorems' LaTeX package.
%%%



\documentclass[11pt,a4paper,oneside,titlepage]{alb-corp}



%
% URL Typesetting
%
% See: 'url.sty'.

\usepackage{url}


%
% The package being documented.

\usepackage{alb-theorems}


%
% Sloppy Line Breaks
%
% Turn off careful line breaks and hyphenation.

\sloppy



\begin{document}



%%%
%%% TITLE
%%%

\albTitle{%
  Typesetting Theorems Under \LaTeX{}%
}

\begin{albTitlePage}

  \albTitlePageSection{Author}

  Andrew Lincoln Burrow\\
  \texttt{albcorp@gmail.com}


  \albTitlePageSection{Abstract}

  The \texttt{AlbTheorems} package provides a single
  \texttt{alb-theorems} \LaTeX{} package to provide markup for a minimal
  collection of theorem like environments.  It provides abbreviated
  cross referencing commands that construct margin notes with page
  numbers, and places the markup in the \albLogo{} namespace.
  Therefore, the package provides a small collection of environments.
  The package is supported by an emacs lisp file customising \AUCTeX{}
  and \RefTeX{}, which provides environment name completion and argument
  prompting, and constructs labels in the file namespace.


  \albTitlePageSection{Copyright}

  Copyright \copyright{} 2001--2006, 2013 Andrew Lincoln Burrow.\\
  This program may be distributed and/or modified under the conditions
  of the \LaTeX{} Project Public License, either version 1.3 of this
  license or (at your option) any later version.

  \medskip{}

  The latest version of this license is in
  \begin{quote}
    \url{http://www.latex-project.org/lppl.txt}
  \end{quote}
  and version 1.3 or later is part of all distributions of LaTeX version
  2005/12/01 or later.

  \medskip{}

  This work has the LPPL maintenance status `author-maintained'.

  \medskip{}

  This work consists of the files
  \begin{quote}
    \begin{flushleft}
      \url{alb-algorithms.sty}, \url{alb-avm.sty}, \url{alb-corp.cls},
      \url{alb-float-tools.sty}, \url{alb-graph-theory.sty},
      \url{alb-journal.cls}, \url{alb-order-theory.sty},
      \url{alb-proofs.sty}, \url{alb-theorems.sty},
      \url{alb-thesis.cls}, \url{alb-algorithms.tex}, \url{alb-avm.tex},
      \url{alb-corp.tex}, \url{alb-float-tools.tex},
      \url{alb-graph-theory.tex}, \url{alb-journal.tex},
      \url{alb-order-theory.tex}, \url{alb-proofs.tex},
      \url{alb-theorems.tex}, \url{alb-thesis.tex}.
      \url{alb-journal-glossary.ist}, \url{alb-journal-index.ist},
      \url{alb-thesis-glossary.ist}, and \url{alb-thesis-index.ist}.
    \end{flushleft}
  \end{quote}


  \albTitlePageSection{Version Information}

  \verb$Revision$\\
  \verb$Date$

\end{albTitlePage}



%%%
%%% INTRODUCTION
%%%

\section{Introduction}
\label{sec:alb-theorems-documentation:intr}

The \texttt{alb-theorems} \LaTeX{} package provides a group of theorem
like environments for typesetting mathematical propositions.  It also
provides a system to construct cross reference material that can be
typeset in the margin.  The package is less likely to be useful to
others, because it encodes decisions about the theorem numbering scheme
and the types of proposition used.

This package aso includes \AUCTeX{} code to override the built-in label
generation rules for the theorem environments.  The system ensures dirty
labels are not reissued.



%%%
%%% USING THE COMMANDS AND ENVIRONMENTS
%%%

\section{Using the Commands and Environments}
\label{sec:alb-theorems-documentation:using-comm-envir}

The environments of \texttt{alb-theorems} are simple.  Therefore, we can
list them with little need for analysis.  All the theorem like
environments share the same counter which is a subcounter to the chapter
counter.

\begin{description}
\item[\albLtxEnv{albDefinition}\albLtxOpt{name}] Create a definition
  environment with an optional name.

  For example,
  \begin{quote}
\begin{verbatim}
\begin{albDefinition}[Example Definition]
  \label{def:alb-theorems-documentation:example-defin}
  This is an example.  $f(x) = x^ 2$.
\end{albDefinition}
\end{verbatim}
  \end{quote}
  produces the following definition.

  \begin{albDefinition}[Example Definition]
    \label{def:alb-theorems-documentation:example-defin}
    This is an example.  $f(x) = x^ 2$.
  \end{albDefinition}

\item[\albLtxEnv{albTheorem}\albLtxOpt{name}] Create a theorem
  environment with an optional name.

  For example,
  \begin{quote}
\begin{verbatim}
\begin{albTheorem}[Theorem of Everything]
  \label{thm:alb-theorems-documentation:theorem-everyth}
  Every thing is something.  Consider the following.
  $f(x) = 3x^2 - 2x + 12$.
\end{verbatim}
  \end{quote}
  produces the following theorem.

  \begin{albTheorem}[Theorem of Everything]
    \label{thm:alb-theorems-documentation:theorem-everyth}
    Every thing is something.  Consider the following.  $f(x) = 3x^2 -
    2x + 12$.
  \end{albTheorem}

\item[\albLtxEnv{albLemma}\albLtxOpt{name}] Create a lemma environment
  with an optional name.

  For example,
  \begin{quote}
\begin{verbatim}
\begin{albLemma}
  \label{lem:alb-theorems-documentation:1}
  Let $n$ be a number such that $n$ does not occur on the
  hand written bill for a hamburger with the lot.  Then $n$
  is unlikely to be the number $7.50$.
\end{albLemma}
\end{verbatim}
  \end{quote}
  produces the following lemma.

  \begin{albLemma}
    \label{lem:alb-theorems-documentation:1}
    Let $n$ be a number such that $n$ does not occur on the hand written
    bill for a hamburger with the lot.  Then $n$ is unlikely to be the
    number $7.50$.
  \end{albLemma}

\item[\albLtxEnv{albRemark}\albLtxOpt{name}] Create a remark environment
  with an optional name.

  For example,
  \begin{quote}
\begin{verbatim}
\begin{albRemark}
  \label{rem:alb-theorems-documentation:2}
  Let $x$ be the price of minimum chips at a fish and chip
  shop in rural Victoria.  Then $2x$ chips at the same shop
  is likely to be too many.
\end{albRemark}
\end{verbatim}
  \end{quote}
  produces the following remark.

  \begin{albRemark}
    \label{rem:alb-theorems-documentation:2}
    Let $x$ be the price of minimum chips at a fish and chip shop in
    rural Victoria.  Then $2x$ chips at the same shop is likely to be
    too many.
  \end{albRemark}

\item[\albLtxEnv{albCorollary}\albLtxOpt{name}] Create corollary
  environment with an optional name.

  For example,
  \begin{quote}
\begin{verbatim}
  \begin{albCorollary}
  \label{cor:alb-theorems-documentation:3}
  It is a bad idea to order $2x$ worth of chips unless you
  are accompanied by another hungry person.
\end{albCorollary}
\end{verbatim}
  \end{quote}
  produces the following corollary.

  \begin{albCorollary}
    \label{cor:alb-theorems-documentation:3}
    It is a bad idea to order $2x$ worth of chips unless you are
    accompanied by another hungry person.
  \end{albCorollary}
\end{description}

There is an additional list environment for making proposition lists
inside a theorem like environment.  It does not support
cros-referencing.  Use equation numbers instead.

\begin{description}
\item[\albLtxEnv{albPropositions}] Enumerate the propositions within a
  theorem-like environment.  The label is guaranteed to be typeset in
  roman rather than italic, as is the mathematical convention.
\end{description}

In addition to the standard cross references generated by the
\albLtxCmd{ref} command, abbreviated cross reference commands are also
provided.  An command for an abbreviated reference exists for each
theorem like environment.

\begin{description}
\item[\albLtxCmd{albDRef}\albLtxArg{label}] Typeset an abbreviated
  reference to the \albLtxPrm{label} definition and record the cross
  reference for the page.  The reference is prefixed by the letter D.

  For example,
  \begin{quote}
\begin{verbatim}
\albDRef{def:alb-theorems-documentation:example-defin}
\end{verbatim}
  \end{quote}
  produces \albDRef{def:alb-theorems-documentation:example-defin}.

\item[\albLtxCmd{albTRef}\albLtxArg{label}] Typeset an abbreviated
  reference to the \albLtxPrm{label} theorem and record the cross
  reference for the page.  The reference is prefixed by the letter T.

  For example,
  \begin{quote}
\begin{verbatim}
\albTRef{thm:alb-theorems-documentation:theorem-everyth}
\end{verbatim}
  \end{quote}
  produces \albTRef{thm:alb-theorems-documentation:theorem-everyth}.

\item[\albLtxCmd{albLRef}\albLtxArg{label}] Typeset an abbreviated
  reference to the \albLtxPrm{label} lemma and record the cross
  reference for the page.  The reference is prefixed by the letter L.

  For example,
  \begin{quote}
\begin{verbatim}
\albLRef{lem:alb-theorems-documentation:1}
\end{verbatim}
  \end{quote}
  produces \albLRef{lem:alb-theorems-documentation:1}.

\item[\albLtxCmd{albRRef}\albLtxArg{label}] Typeset an abbreviated
  reference to the \albLtxPrm{label} remark and record the cross
  reference for the page.  The reference is prefixed by the letter R.

  For example,
  \begin{quote}
\begin{verbatim}
\albRRef{lem:alb-theorems-documentation:2}
\end{verbatim}
  \end{quote}
  produces \albRRef{rem:alb-theorems-documentation:2}.

\item[\albLtxCmd{albCRef}\albLtxArg{label}] Typeset an abbreviated
  reference to the \albLtxPrm{label} corollary and record the cross
  reference for the page.  The reference is prefixed by the letter C.

  For example,
  \begin{quote}
\begin{verbatim}
\albCRef{cor:alb-theorems-documentation:3}
\end{verbatim}
  \end{quote}
  produces \albCRef{cor:alb-theorems-documentation:3}.
\end{description}



%%%
%%% THEOREM REFERENCE LISTS
%%%

\section{Theorem Reference Lists}
\label{sec:alb-theorems-documentation:theorem-refer-lists}

As noted above, the abbreviated reference commands have a second effect
of accumulating the list of issued references.  This list can be used in
page layouts, such as \texttt{alb-thesis}, to display the most recently
referenced theorems and their page numbers.  These internal commands are
briefly described here.

\begin{description}
\item[\albLtxCmd{alb@ExpandTheoremRefList}] Expands the theorem
  reference list into an index for the references.

  For example,
  \begin{quote}
\begin{verbatim}
\begin{quote}
  \makeatletter
  \alb@ExpandTheoremRefList
  \makeatother
\end{quote}
\end{verbatim}
  \end{quote}
  produces
  \begin{quote}
    \makeatletter%
    \alb@ExpandTheoremRefList%
    \makeatother%
  \end{quote}

  If this list is generated per page, then care must be taken to reset
  the list on each page with the following code.
  \begin{quote}
\begin{verbatim}
\makeatletter
\let\alb@TheoremRefList=\relax
\makeatother
\end{verbatim}
  \end{quote}
\end{description}



%%%
%%% AUCTEX CUSTOMISATIONS
%%%

\section{\AUCTeX\ Customisations}
\label{sec:alb-theorems-documentation:auctex-cust}

Under \AUCTeX{} the file \texttt{alb-theorems.el} is automatically
loaded whenever the \texttt{alb-theorems} package is used.  The
customisation adds the theorem environmens to \AUCTeX{}.  This provides
the simple prompting for all the supplied environments.

In addition, \texttt{alb-theorems.el} causes a special theorem numbering
counter to be stored as a local variable.
\texttt{alb-LaTeX-theorem-counter} counts numbers assigned to theorem
environments.  This ensures dirty numbers are not reissued, as reissuing
a number could make stale references hard to detect.



%%%
%%% MAKEFILE TARGETS
%%%

\section{Makefile Targets}
\label{sec:alb-theorems-documentation:makef-targ}

The \texttt{AlbLaTeXDocumentTemplate} makefile provides a target to
relabel the theorem-like environments of this package.  The
\texttt{alb-relabel-thm} target edits the \LaTeX{} source in an attempt
to match theorem labels to theorem numbers.  Labels of the form
\albLtxPrm{type}\texttt{:}\albLtxPrm{identifier}\texttt{:}\albLtxPrm{number}
are processed, where \albLtxPrm{type} is one of \texttt{thm},
\texttt{lem}, \texttt{rem}, or \texttt{cor}. \albLtxPrm{number} is
rewritten to the last part of the theorem number.  This is helpful in
proof reading.



\end{document}



%%% Local Variables:
%%% mode: latex
%%% TeX-master: t
%%% alb-LaTeX-theorem-counter: 3
%%% End:
