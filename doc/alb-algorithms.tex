%%%
%%% AlbAlgorithms/alb-algorithms-documentation.tex
%%%
%%%     See copyright notice and license in text.
%%%
%%%   - Documentation for the 'alb-algorithms' LaTeX package.
%%%



\documentclass[11pt,a4paper,oneside,titlepage]{alb-corp}



%
% URL Typesetting
%
% See: 'url.sty'.

\usepackage{url}


%
% Base AMS-LaTeX package.
%
% See: 'amsldoc.dvi'

\usepackage{amsmath}


%
% The package being documented.

\usepackage{alb-algorithms}


%
% Sloppy Line Breaks
%
% Turn off careful line breaks and hyphenation.

\sloppy



\begin{document}



%%%
%%% TITLE
%%%

\albTitle{%
  Typesetting Algorithms Under \LaTeX{}%
}

\begin{albTitlePage}

  \albTitlePageSection{Author}

  Andrew Lincoln Burrow\\
  \url{albcorp@gmail.com}


  \albTitlePageSection{Abstract}

  The \texttt{AlbAlgorithms} package provides a single
  \texttt{alb-algorithms} \LaTeX{} package to typeset algorithms.  While
  the \texttt{algorithmic} package is already available,
  \texttt{alb-algorithms} is designed to rectify certain deficiencies.
  In particular, it improves line numbering for groups of procedures in
  an algorithm, matches the markup structure to the algorithm's block
  structure, supports cross-referencing with \AUCTeX{} and \RefTeX{},
  and uses the \albLogo{} namespace.  The package is supported by an
  emacs lisp file customising \AUCTeX{} and \RefTeX{} which automates
  the markup process of algorithm floats and block structure, and
  constructs labels in the file namespace.


  \albTitlePageSection{Copyright}

  Copyright \copyright{} 2001--2006 Andrew Lincoln Burrow.\\
  This program may be distributed and/or modified under the conditions
  of the \LaTeX{} Project Public License, either version 1.3 of this
  license or (at your option) any later version.

  \medskip{}

  The latest version of this license is in
  \begin{quote}
    \url{http://www.latex-project.org/lppl.txt}
  \end{quote}
  and version 1.3 or later is part of all distributions of LaTeX version
  2005/12/01 or later.

  \medskip{}

  This work has the LPPL maintenance status `author-maintained'.

  \medskip{}

  This work consists of the files
  \begin{quote}
    \begin{flushleft}
      \url{alb-algorithms.sty}, and
      \url{alb-algorithms-documentation.tex}.
    \end{flushleft}
  \end{quote}


  \albTitlePageSection{Version Information}

  \verb$Revision$\\
  \verb$Date$

\end{albTitlePage}



%%%
%%% INTRODUCTION
%%%

\section{Introduction}
\label{sec:alb-algorithms-documentation:intr}

The \texttt{alb-algorithms} \LaTeX{} package is designed to typeset
algorithms.  This system necessarily promotes a certain view of
algorithms and data structures.  In this case, algorithms are modelled
by side effect producing functions termed \emph{procedures}, and data
structures are modelled by \emph{accessor} functions.

In typographic terms, \texttt{alb-algorithms} extends special treatment
to algorithms with respect to page layout and notation.  In terms of
page layout, algorithms are handled like figures.  Namely, they are
typeset as floats so that algorithms appear at the top of pages, and no
algorithm is split across pages.  While this means that an algorithm
will occasionally be located a few pages from the accompanying text, it
ensures each algorithm can be studied as a textual unit free of page
breaks.  In terms of notation, algorithms are subject to a number of
conventions intended to improve their readability and to highlight
certain facts.  An algorithm is presented as a function prototype
followed by a line numbered body.  The function prototype is typeset in
typewriter-style characters, as are calls to all other functions also
defined by algorithms.  The remainder of the text in the body is
structured around keywords typeset in boldface, and typeset mathematics.
The body is divided into lines which are numbered, and these numbers are
used to reference details in the algorithm.  Therefore, the line numbers
are made unique within a float by appending a letter --- a new letter
for each procedure in the algorithm.  These considerations lead to a
collection of \LaTeX{} commands and environments.

The \texttt{alb-algorithms} package provides:
\begin{itemize}
\item a float environment \texttt{algorithm} for algorithms, and a
  command \texttt{listofalgorithms} for generating front matter;
\item commands \texttt{albNewAccessorIdent} and
  \texttt{albNewProcedureIdent} to declare new \LaTeX{} commands to
  typeset accessors and procedures respectively, both as identifiers and
  as prototypes;
\item master environments \texttt{albAlgorithmic} and
  \texttt{albAlgorithmic*} to typeset block structured pseudo-code where
  block structure is indicated typographically by indentation, and in
  mark up by the \texttt{albBlock} environment; and
\item commands \texttt{albLet}, \texttt{albAssign}, \texttt{albWhile},
  \texttt{albForAll}, \texttt{albForSequence}, \texttt{albIf},
  \texttt{albElseIf}, \texttt{albElse}, and \texttt{albReturn} commands
  to typeset the keywords of both simple and compound statements.
\end{itemize}

An important feature of \texttt{alb-algorithms} is the support for
labels.  This makes it possible to refer to exact line numbers in the
analysis of algorithms.  Each line in an algorithm corresponds to an
\texttt{item} in either an \texttt{albAlgorithmic} environment or a
nested \texttt{albBlock} environment, and each such \texttt{item} can be
referred to via the \LaTeX{} \texttt{label} and \texttt{ref} mechanism.
For this reason, the \AUCTeX{} extensions of \texttt{alb-algorithms}
provide support for automatically generating useful labels and providing
useful context for labels during cross-referencing.



%%%
%%% USING THE COMMANDS AND ENVIRONMENTS
%%%

\section{Using the Commands and Environments}
\label{sec:alb-algorithms-documentation:using-comm-envir}

The environments of \texttt{alb-algorithms} place certain syntactic
restrictions on their use.  This section considers these restrictions
and the typographic meaning of the markup.
Section~\ref{sec:alb-algorithms-documentation:worked-example} provides a
worked example.
Section~\ref{sec:alb-algorithms-documentation:auctex-cust} describes how
the \AUCTeX{} customisation eases the burden of entering syntactically
correct \LaTeX{} code.



\subsection{List of Algorithms}
\label{sec:alb-algorithms-documentation:list-algor}

The \texttt{alb-algorithms} package provides a new command to generate a
list of algorithms for inclusion with the table of contents.  Otherwise,
this command is analogous to the \texttt{listoffigures} command.

\begin{description}
\item[\albLtxCmd{listofalgorithms}] Generate a list of the figure float
  environments in the document.  The command is analogous to the
  built-in LaTeX command \albLtxCmd{listoffigures}.
\end{description}



\subsection{Algorithm Floats}
\label{sec:alb-algorithms-documentation:algor-floats}

The \texttt{alb-algorithms} package provides the \texttt{algorithm}
environment to present an algorithm as a single, unbroken typographic
structure.  The most visible benefit is that \AUCTeX{} is automatically
extended to provide improved prompting, so that the identifier in the
first prototype is translated into the float's label.

\begin{description}
\item[\albLtxEnv{algorithm}] Float an algorithm.  The procedure
  prototype should be the first typeset material.  See
  Section~\ref{sec:alb-algorithms-documentation:worked-example} for a
  worked example.
\end{description}



\subsection{Accessor and Procedure Definition}
\label{sec:alb-algorithms-documentation:accessor-proc-defin}

The \texttt{alb-algorithms} \LaTeX{} package supports the typesetting of
the accessor and procedure functions that define algorithms.  To this
end it provides two commands that create new commands:
\texttt{albNewAccessorIdent} creates new accessor functions, and
\texttt{albNewProcedureIdent} creates new procedures.  Again, \AUCTeX{}
is extended to prompt input of these commands.  However, it is further
extended to parse these commands so that the newly defined accessor and
procedure typesetting commands are also available to tab completion.

\begin{description}
\item[%
  \albLtxCmd{albNewAccessorIdent}%
  \albLtxArg{identcmd}\albLtxArg{protocmd}%
  \albLtxArg{name}\albLtxArg{args}%
  ] Create two \LaTeX{} commands called \albLtxPrm{identcmd} and
  \albLtxPrm{callcmd} to respectively typeset an accessor's identifier
  and prototype.  The typeset identifier is \albLtxPrm{name} and the
  number of arguments in the prototype is \albLtxPrm{args}.

  For example,
  \begin{quote}
\begin{verbatim}
\albNewAccessorIdent%
  {accFeaturesId}{\accFeatures}%
  {features}{2}
\end{verbatim}
  \end{quote}
  produces the command \albLtxCmd{accFeaturesId} to typeset the
  accessor's identifer \texttt{features} and the command
  \albLtxCmd{accFeatures}\albLtxArg{arg1}\albLtxArg{arg2} to typeset a
  call to the accessor \texttt{features}(\albLtxPrm{arg1},
  \albLtxPrm{arg2}).

  The \albLtxPrm{callcmd} command must only be used in math mode, while
  the \albLtxPrm{identcmd} command may be used outside of math mode.
  Outside of math mode, special consideration is given to hyphens in
  \albLtxPrm{name}.  Namely, \albLtxPrm{name} may contain hyphens.
  These will be typeset as hyphens rather than minuses, and will be used
  in the \LaTeX{} line breaking algorithm when \albLtxPrm{identcmd} is
  used outside of math mode.  The use of hyphens in names is in keeping
  with practice in languages like lisp, and improves the line breaking
  in text containing long identifiers.

\item[%
  \albLtxCmd{albNewProcedureIdent}%
  \albLtxArg{identcmd}\albLtxArg{protocmd}%
  \albLtxArg{name}\albLtxArg{args}%
  ] Create two \LaTeX{} commands \albLtxPrm{identcmd} and
  \albLtxPrm{callcmd} to respectively typeset a procedure identifier and
  prototype.  The typographical meaning is identical to that of
  \texttt{albNewAccessorIdent}.  The markup distinction allows \AUCTeX{}
  customisation to support the semantic difference in the denoted
  objects.
\end{description}



\subsection{Block Structured Algorithms}
\label{sec:alb-algorithms-documentation:block-struct-algor}

The final piece in \texttt{alb-algorithms} is the support for
typesetting line numbered, block structured pseudo-code.  This support
is provided by nesting the environments \texttt{algorithm},
\texttt{albAlgorithmic}, and \texttt{albBlock}.  This section describes
the syntax and typographical meaning of these structures.

An \texttt{algorithm} float contains one or more \texttt{albAlgorithmic}
or \texttt{albAlgorithmic*} environments.  These correspond to the
procedure definitions comprising the algorithm.  Behind the scenes the a
counter is kept for the number of procedures in the float, so that line
numbers are prefixed by an alphabetic sequence letter.  Line numbers in
the each procedure are prefixed by an uppercase letter.

An instance of the \texttt{albAlgorithmic} list environment typesets a
line per \texttt{item} command.  However, some items may be nested
within parasitical \texttt{albBlock} list environments.  These
environments do not contain any of their own counters, but simply indent
the nested items.  Thus, \texttt{albBlock} environments correspond to
blocks in a block structured language.  The pseudo-code keywords
introducing blocks and mathematical entities in a line are typeset by
separate \LaTeX{} commands.

This structure provides the following benefits.
\begin{itemize}
\item Line numbers are named apart into procedures when an algorithm is
  implemented by a number of procedures.
\item The indentation in the \LaTeX{} source matches the indentation in
  the typeset algorithm.
\item The \texttt{albAlgorithmic} and \texttt{albBlock} list environments
  do not disguise their \texttt{item} and \texttt{label} components,
  making AUCTeX and RefTeX cross-referencing support much more
  effective.
\item The system is extensible to different keywords without involving
  the structural aspects.
\end{itemize}

\begin{description}
\item[\albLtxEnv{albAlgorithmic} and \albLtxEnv{albAlgorithmic*}] The
  outermost list environment for statements in a procedure.  Each
  \texttt{item} is associated with a statement, and may have a
  \texttt{label} for cross-referencing of the statement.  Compound
  statements introduce nested \texttt{albBlock} list environments.

  The un-starred version must occur within an \texttt{algorithm}
  environment.  The starred version is used outside the
  \texttt{albAlgorithmFloat} environment, and does not implement
  procedure numbering in the line numbering scheme.

\item[\albLtxEnv{albBlock}] A list environment for nesting blocks of
  lines associated with compound statements.  Each \texttt{item} is
  associated with a statement, and may have a \texttt{label} for
  cross-referencing of the statement.  Compound statements introduce
  nested \texttt{albBlock} list environments.

  The environment must occur within an \texttt{albAlgorithmic} or
  \texttt{albAlgorithmic*} environment.

\item[\albLtxCmd{albLet}\albLtxArg{var}\albLtxArg{val}] Command to
  typeset a simple statement introducing the variable typeset by
  \albLtxPrm{var} and assigning it the value typeset by \albLtxPrm{val}.
  For example,
  \begin{quote}
\begin{verbatim}
\albLet{$A \subseteq X$}{$\emptyset$}
\end{verbatim}
  \end{quote}
  produces \albLet{$A \subseteq X$}{$\emptyset$}.

\item[\albLtxCmd{albAssign}\albLtxArg{var}\albLtxArg{val}] Command to
  typeset a simple statement assigning the value typeset by
  \albLtxPrm{val} to variable typeset by \albLtxPrm{var}.  For example,
  \begin{quote}
\begin{verbatim}
\albAssign{$A$}{$A \cup \{x\}$}
\end{verbatim}
  \end{quote}
  produces \albAssign{$A$}{$A \cup \{x\}$}.

\item[\albLtxCmd{albReturn}\albLtxArg{val}] Command to typeset a simple
  statement returning the value typeset by \albLtxPrm{val}.  For
  example,
  \begin{quote}
\begin{verbatim}
\albReturn{$A \cup \{x\}$}
\end{verbatim}
  \end{quote}
  produces \albReturn{$A \cup \{x\}$}.

\item[\albLtxCmd{albWhile}\albLtxArg{cond}] Command to typeset the test
  part of a while statement.  The condition is typeset by
  \albLtxPrm{cond}.  It should be followed by an \texttt{albBlock}
  environment listing the statements in the body of the loop.  For
  example,
  \begin{quote}
\begin{verbatim}
\albWhile{$A \neq \emptyset$}
\end{verbatim}
  \end{quote}
  produces \albWhile{$A \neq \emptyset$}.

\item[\albLtxCmd{albForAll}\albLtxArg{var}\albLtxArg{set}] Command to
  typeset the loop definition part of a for statement where the order is
  undetermined.  The loop variable is typeset by \albLtxPrm{var}, and
  the collection is typeset by \albLtxPrm{set}.  It should be followed
  by an \texttt{albBlock} environment listing the statements in the body
  of the loop.  For example,
  \begin{quote}
\begin{verbatim}
\albForAll{$x$}{$A$}
\end{verbatim}
  \end{quote}
  produces \albForAll{$x$}{$A$}.

\item[\albLtxCmd{albForSequence}\albLtxArg{var}\albLtxArg{seq}] Command
  to typeset the loop definition part of a for statement where the order
  is explicit.  The loop variable is typeset by \albLtxPrm{var}, and the
  sequence is typeset by \albLtxPrm{seq}.  It should be followed by an
  \texttt{albBlock} environment listing the statements in the body of
  the loop.  For example,
  \begin{quote}
\begin{verbatim}
\albForSequence{$x$}{$0, 1, \ldots n$}
\end{verbatim}
  \end{quote}
  produces \albForSequence{$x$}{$0, 1, \ldots n$}.

\item[%
  \albLtxCmd{albIf}\albLtxArg{cond} and
  \albLtxCmd{albElseIf}\albLtxArg{cond} and \albLtxCmd{albElse}%
  ] Commands to typeset if statements.  The condition is typeset by
  \albLtxPrm{cond}.  Each form should be followed by an
  \texttt{albBlock} environment listing the statements in the body.  For
  example,
  \begin{quote}
\begin{verbatim}
\albElseIf{$i \neq 0$}
\end{verbatim}
  \end{quote}
  produces \albElseIf{$i \neq 0$}.
\end{description}



%%%
%%% WORKED EXAMPLE
%%%

\section{Worked Example}
\label{sec:alb-algorithms-documentation:worked-example}

This section contains a worked example.  The algorithm is a depth-first
traversal over a typed feature structure.  It is depicted in
Algorithm~\ref{alg:depth-first-search}.

The first step is to define the accessor functions used to describe the
data structure.  The \texttt{albNewAccessorIdent} command is used.
\AUCTeX{} customisation simplifies this process for the case where
identifiers are constructed from hyphenated phrases.

\begin{quote}
\begin{verbatim}
\albNewAccessorIdent{\accRootId}{\accRoot}{root}{1}
\albNewAccessorIdent{\accTypeId}{\accType}{type}{1}
\albNewAccessorIdent{\accValueId}{\accValue}{value}{3}
\end{verbatim}
\end{quote}

\albNewAccessorIdent{\accRootId}{\accRoot}{root}{1}
\albNewAccessorIdent{\accTypeId}{\accType}{type}{1}
\albNewAccessorIdent{\accFeaturesId}{\accFeatures}{features}{2}
\albNewAccessorIdent{\accValueId}{\accValue}{value}{3}
\albNewAccessorIdent{\accPathValueId}{\accPathValue}{path-value}{2}

Next, define the functions that comprise the algorithm.  In this
example, the algorithm is implemented by two functions.  The first
function declares a new variable to hold the result and initialises this
variable.  It then calls the second function, which performs the
depth-first recursive traversal of the typed feature structure.

\begin{quote}
\begin{verbatim}
\albNewProcedureIdent%
{\prcDepthFirstSearchId}{\prcDepthFirstSearch}%
{Depth-First-Search}{5}
\albNewProcedureIdent%
{\prcDepthFirstSearchVisitId}{\prcDepthFirstSearchVisit}%
{Depth-First-Search-Visit}{6}
\end{verbatim}
\end{quote}

\albNewProcedureIdent%
{\prcDepthFirstSearchId}{\prcDepthFirstSearch}%
{Depth-First-Search}{5}
\albNewProcedureIdent%
{\prcDepthFirstSearchVisitId}{\prcDepthFirstSearchVisit}%
{Depth-First-Search-Visit}{6}

The algorithm is also parameterised on the actions to take at various
stages in the traversal.  Therefore, we must also define names for these
functions in the parameter list.  This is achieved as for the names of
the functions implementing the traversal.

\begin{quote}
\begin{verbatim}
\albNewProcedureIdent{\prcPreId}{\prcPre}{Pre}{2}
\albNewProcedureIdent{\prcPostId}{\prcPost}{Post}{2}
\albNewProcedureIdent{\prcFwdId}{\prcFwd}{Fwd}{3}
\albNewProcedureIdent{\prcBckId}{\prcBck}{Bck}{3}
\end{verbatim}
\end{quote}

\albNewProcedureIdent{\prcPreId}{\prcPre}{Pre}{2}
\albNewProcedureIdent{\prcPostId}{\prcPost}{Post}{2}
\albNewProcedureIdent{\prcFwdId}{\prcFwd}{Fwd}{3}
\albNewProcedureIdent{\prcBckId}{\prcBck}{Bck}{3}

Algorithm~\ref{alg:depth-first-search} shows the constructed algorithm,
and Figure~\ref{fig:alb-algorithms-documentation:source-code-prcd}
contains the souce code used to generate the float.

% ----------------------------------------------------------------------
%                                          Algorithm: Depth-First-Search
%
\begin{algorithm}
  $\prcDepthFirstSearch%
  {F}{\prcPreId}{\prcFwdId}{\prcBckId}{\prcPostId}$
  \begin{albAlgorithmic}
  \item \label{ln:alb-algorithms-documentation:depth-first-search-1} %
    \albLet{$Q \subseteq \mathsf{Node}$}{$\emptyset$}

  \item \label{ln:alb-algorithms-documentation:depth-first-search-2} %
    $\prcDepthFirstSearchVisit%
    {F}{\prcPreId}{\prcFwdId}{\prcBckId}{\prcPostId}%
    {\accRoot{F}}$
  \end{albAlgorithmic}

  \medskip{}

  $\prcDepthFirstSearchVisit%
  {F}{\prcPreId}{\prcFwdId}{\prcBckId}{\prcPostId}{q}$
  \begin{albAlgorithmic}
  \item \label{ln:alb-algorithms-documentation:depth-first-search-3} %
    $\prcPre{F}{q}$

  \item \label{ln:alb-algorithms-documentation:depth-first-search-4} %
    \albAssign{$Q$}{$Q \cup \{q\}$}

  \item \label{ln:alb-algorithms-documentation:depth-first-search-5} %
    \albForAll{$f$}{$\accFeatures{F}{q}$}

    \begin{albBlock}
    \item \label{ln:alb-algorithms-documentation:depth-first-search-6} %
      $\prcFwd{F}{f}{q}$

    \item \label{ln:alb-algorithms-documentation:depth-first-search-7} %
      \albIf{$\accValue{F}{f}{q} \not\in Q$}

      \begin{albBlock}
      \item \label{ln:alb-algorithms-documentation:depth-first-search-8} %
        $\prcDepthFirstSearchVisit%
        {F}{\prcPreId}{\prcFwdId}{\prcBckId}{\prcPostId}%
        {\accValue{F}{f}{q}}$
      \end{albBlock}

    \item \label{ln:alb-algorithms-documentation:depth-first-search-9} %
      $\prcBck{F}{f}{q}$
    \end{albBlock}

  \item \label{ln:alb-algorithms-documentation:depth-first-search-10} %
    $\prcPost{F}{q}$
  \end{albAlgorithmic}
  \caption{Depth first search in a typed feature structure.}
  \label{alg:depth-first-search}
\end{algorithm}
%
%                                          Algorithm: Depth-First-Search
% ----------------------------------------------------------------------

\begin{figure}
  \small{}
\begin{verbatim}
\begin{algorithm}
  $\prcDepthFirstSearch%
  {F}{\prcPreId}{\prcFwdId}{\prcBckId}{\prcPostId}$
  \begin{albAlgorithmic}
  \item \label{ln:depth-first-search-1}
    \albLet{$Q \subseteq \mathsf{Node}$}{$\emptyset$}

  \item \label{ln:depth-first-search-2}
    $\prcDepthFirstSearchVisit%
    {F}{\prcPreId}{\prcFwdId}{\prcBckId}{\prcPostId}%
    {\accRoot{F}}$
  \end{albAlgorithmic}

  \medskip{}

  $\prcDepthFirstSearchVisit%
  {F}{\prcPreId}{\prcFwdId}{\prcBckId}{\prcPostId}{q}$
  \begin{albAlgorithmic}
  \item \label{ln:depth-first-search-visit-1}
    $\prcPre{F}{q}$

  \item \label{ln:depth-first-search-visit-2}
    \albAssign{$Q$}{$Q \cup \{q\}$}

  \item \label{ln:depth-first-search-visit-3}
    \albForAll{$f$}{$\accFeatures{F}{q}$}

    \begin{albBlock}
    \item \label{ln:depth-first-search-visit-4}
      $\prcFwd{F}{f}{q}$

    \item \label{ln:depth-first-search-visit-5}
      \albIf{$\accValue{F}{f}{q} \not\in Q$}

      \begin{albBlock}
      \item \label{ln:depth-first-search-visit-6}
        $\prcDepthFirstSearchVisit%
        {F}{\prcPreId}{\prcFwdId}{\prcBckId}{\prcPostId}%
        {\accValue{F}{f}{q}}$
      \end{albBlock}

    \item \label{ln:depth-first-search-visit-7}
      $\prcBck{F}{f}{q}$
    \end{albBlock}

  \item \label{ln:depth-first-search-visit-8}
    $\prcPost{F}{q}$
  \end{albAlgorithmic}
  \caption{Depth first search in a typed feature structure.}
  \label{alg:depth-first-search}
\end{algorithm}
\end{verbatim}
  \caption{Source code to \prcDepthFirstSearchId{}}
  \label{fig:alb-algorithms-documentation:source-code-prcd}
\end{figure}



%%%
%%% AUCTEX CUSTOMISATIONS
%%%

\section{\AUCTeX\ Customisations}
\label{sec:alb-algorithms-documentation:auctex-cust}

Under \AUCTeX{} the file \texttt{alb-algorithms.el} is automatically
loaded whenever the \texttt{alb-algorithms} package is used.  The
customisation adds support for the commands and environments from
\texttt{alb-algorithms}.  In addition, the \texttt{reftex-label} command
(\texttt{C-c (}) constructs labels that indicate the containing
procedure, and the \texttt{reftex-reference} command (\texttt{C-c )})
determines that a line number is being referred to when the preceding
text is \texttt{Line} or \texttt{Lines} and formats the context lines to
better indicate the typeset output.



%%%
%%% MAKEFILE TARGETS
%%%

\section{Makefile Targets}
\label{sec:alb-algorithms-documentation:makef-targ}

The \texttt{AlbLaTeXDocumentTemplate} makefile provides a target to
relabel the lines in the algorithm environment of this package.  The
\texttt{alb-relabel-ln} target edits the \LaTeX{} source in an attempt
to match theorem labels to theorem numbers.  Labels of the form
\texttt{ln:}\albLtxPrm{identifier}\texttt{:}\albLtxPrm{identifier}%
\texttt{:}\albLtxPrm{line}
\texttt{ln:}\albLtxPrm{identifier}\texttt{:}\albLtxPrm{identifier}%
\texttt{-}\albLtxPrm{line} are processed, where \albLtxPrm{line} is a
number or lowercase letter followed by a number.  \albLtxPrm{line} is
rewritten to the line number, and the preceeding separator is rewritten
to a colon.  This is helpful in proof reading.



\end{document}



%%% Local Variables:
%%% mode: latex
%%% TeX-master: t
%%% alb-LaTeX-line-counter: 11
%%% End:
