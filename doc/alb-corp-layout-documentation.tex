%%%
%%% AlbCorpLayout/alb-corp-layout-documentation.tex
%%%
%%%     See copyright notice and license in text.
%%%
%%%   - Documentation for the 'alb-corp' LaTeX document class.
%%%



\documentclass[11pt,a4paper,oneside,titlepage]{alb-corp}



%
% URL Typesetting
%
% See: 'url.sty'.

\usepackage{url}


%
% Sloppy Line Breaks
%
% Turn off careful line breaks and hyphenation.

\sloppy



\begin{document}


%%%
%%% TITLE
%%%

\albTitle{%
  Page Layout for \albLogo{}Corp Documents%
}


\begin{albTitlePage}

  \albTitlePageSection{Author}

  Andrew Lincoln Burrow\\
  \url{albcorp@gmail.com}


  \albTitlePageSection{Abstract}

  The \texttt{AlbCorpLayout} package provides a single \texttt{alb-corp}
  \LaTeX{} document class.  The class implements a book design for
  \albLogo{}Corp documents that is designed for the documentation of
  \LaTeX{} packages.  It sets up the page layout and section formatting.
  It also provides a command to capture the document title, commands and
  environments to typeset the title material, commands to typeset
  \LaTeX{} commands, a collection of logo commands, and markup in the
  \albLogo{} name space.  All commands and environments are available
  through the \AUCTeX{} interface.


  \albTitlePageSection{Copyright}

  Copyright \copyright{} 2002--2006 Andrew Lincoln Burrow.\\
  This program may be distributed and/or modified under the conditions
  of the \LaTeX{} Project Public License, either version 1.3 of this
  license or (at your option) any later version.

  \medskip{}

  The latest version of this license is in
  \begin{quote}
    \url{http://www.latex-project.org/lppl.txt}
  \end{quote}
  and version 1.3 or later is part of all distributions of LaTeX version
  2005/12/01 or later.

  \medskip{}

  This work has the LPPL maintenance status `author-maintained'.

  \medskip{}

  This work consists of the files
  \begin{quote}
    \begin{flushleft}
      \url{alb-corp.cls} and \url{alb-corp-layout-documentation.tex}.
    \end{flushleft}
  \end{quote}


  \albTitlePageSection{Version Information}

  \verb$Revision$\\
  \verb$Date$

\end{albTitlePage}



%%%
%%% INTRODUCTION
%%%

\section{Introduction}
\label{sec:alb-corp-layout-examples:intr}

The \texttt{AlbCorpLayout} package is designed to assist in the
documentation of \LaTeX{} packages.  It consists of the document class
\texttt{alb-corp}.  This class is designed to achieve consistency and to
provide guidance when writing documentation for \albLogo{} packages.
The fictional \albLogo{} corporation is mnemonic for the naming
conventions that prefix each identifier by \texttt{alb}.  The
\texttt{alb-corp} class extends this fiction to the presentation of
documentation for packages written by the author.

The page layout provides a wide gutter margin for binding, and a narrow
column close to the outer edge.  Large floats access the inner gutter
via the \texttt{albInflate} environment of the \texttt{AlbFloatTools}
package.

Page numbers are placed in the running headers with the title and
section information.  In twoside mode, the left header contains the page
number and the document title, while the right header contains the
current section title and page number.

In addition to setting the document appearance, the class provides a
single environment for creating a title page, an associated command for
designating the title, another associated command for adding sections to
the titlepage, simple commands for typesetting \LaTeX{} commands, and a
collection of logo commands.  The class is supported by an emacs lisp
file customising \AUCTeX{} to automate the insertion of these commands
and environments.



%%%
%%% SUGGESTED DOCUMENT STRUCTURE
%%%

\section{Suggested Document Structure}
\label{sec:alb-corp-layout-documentation:sugg-docum-struct}

The \texttt{alb-corp} document class places few restrictions on the
document structure.  It is primarily intended for articles.  Documents
must insert title information at the head of the document, since this
information is used in the header.



\subsection{Page Layout Options}
\label{sec:alb-corp-layout-documentation:page-layout-opti}

The \texttt{alb-corp} document class attempts to respect page layout
options.  In particular, you can use \texttt{oneside}, \texttt{twoside},
\texttt{titlepage}, and \texttt{notitlepage} as global options.
However, the page layout will not accomodate \texttt{twocolumn} or
\texttt{reversemp}.  The page layout accomodates \texttt{a4paper} and
\texttt{letterpaper} page sizes.

Page layout options must be placed in the \texttt{documentclass}
declaration, and each relevant option should be explicitly declared
since the \texttt{AlbLaTeXDocumentTemplate} makefile parses document
class options.  The following content is a typical example of the
document class declaration.
\begin{quote}
\begin{verbatim}
\documentclass[10pt,a4paper,twoside]{alb-journal}
\end{verbatim}
\end{quote}



%%%
%%% USING THE COMMANDS AND ENVIRONMENTS
%%%

\section{Using the Commands and Environments}
\label{sec:alb-corp-layout-documentation:using-comm-envir}

The commands and environments of the \texttt{alb-corp} class are divided
into three groups.  The first concerns the title page and title
information.  These commands should occur early in the document.  The
second group concerns the commands to typeset \LaTeX{} commands and
environments.  The third group provides additional logos that are likely
to be referred to whilst documenting \LaTeX{} packages.



\subsection{Title Material}
\label{sec:alb-corp-layout-documentation:title-mater}

Documents of the \texttt{alb-corp} class should begin with an
\texttt{albTitle} command.  This command accepts a single argument,
which is the title of the article.  The value is shared by the page
headers as well as the title page.

Documents should follow the \texttt{albTitle} command with an
\texttt{albTitlePage} environment.  The contents of this environment
should be sectioned by instances of the \texttt{albTitlePageSection}
command, where each instance takes the name of the section as its
argument, and is followed by the text of the section.

For example, a document might begin with the following \LaTeX{} code.
\begin{quote}
\begin{verbatim}
\albTitle{Example Title}

\begin{albTitlePage}
  \albTitlePageSection{Author}

  Andrew Lincoln Burrow\\
  \texttt{albcorp@gmail.com}\\
  \albLogo{} Corp

  \albTitlePageSection{Abstract}

  This document is completely fictitious.
  If you are reading this, then you should stop now.
\end{albTitlePage}
\end{verbatim}
\end{quote}



\subsection{Command Typesetting Commands}
\label{sec:alb-corp-layout-documentation:comm-types-comm}

The next group of commands provided by the \texttt{alb-corp} class
typeset \LaTeX{} commands.  In particular, they provide a simple markup
scheme for the parts of a \LaTeX{} command or environment invocation so
that it can be documented.

\begin{description}
\item[\albLtxCmd{albLtxCmd}\albLtxArg{cmd}] Typeset the \LaTeX{} command
  \albLtxPrm{cmd} with the preceeding backslash.  For example,
  \begin{quote}
\begin{verbatim}
\albLtxCmd{albLogo}
\end{verbatim}
  \end{quote}
  produces \albLtxCmd{albLogo}.

\item[\albLtxCmd{albLtxEnv}\albLtxArg{env}] Typeset the \LaTeX{}
  environment \albLtxPrm{env} with the \texttt{begin} command that would
  open the environment.  For example,
  \begin{quote}
\begin{verbatim}
\albLtxEnv{albTitlePage}
\end{verbatim}
  \end{quote}
  produces \albLtxEnv{albTitlePage}

\item[%
  \albLtxCmd{albLtxOpt}\albLtxArg{opt} and
  \albLtxCmd{albLtxArg}\albLtxArg{arg}%
  ] Typeset an optional argument \albLtxPrm{opt} or a mandatory argument
  \albLtxPrm{arg} supplied to a \LaTeX{} command or environment, and use
  the correct brackets.  For example,
  \begin{quote}
\begin{verbatim}
\albLtxCmd{rule}%
\albLtxOpt{lift}\albLtxArg{width}\albLtxArg{height}
\end{verbatim}
  \end{quote}
  produces
  \albLtxCmd{rule}\albLtxOpt{lift}\albLtxArg{width}\albLtxArg{height}.

\item[\albLtxCmd{albLtxPrm}\albLtxArg{arg}] Typeset the formal parameter
  of an optional or mandatory argument without the brackets so that it
  can be used in the explanatory text.  For
  example,
  \begin{quote}
\begin{verbatim}
\albLtxPrm{arg}
\end{verbatim}
  \end{quote}
  produces \albLtxPrm{arg}.
\end{description}



\subsection{Logo Commands}
\label{sec:alb-corp-layout-documentation:logo-comm}

The remainder of the commands provided by the \texttt{alb-corp} class
typeset logos.

\begin{description}
\item[\albLtxCmd{PDFLaTeX}] produces \PDFLaTeX{}.

\item[\albLtxCmd{AUCTeX}] produces \AUCTeX{}.

\item[\albLtxCmd{RefTeX}] produces \RefTeX{}.

\item[\albLtxCmd{albLogo}] produces \albLogo{}.
\end{description}



%%%
%%% AUCTEX CUSTOMISATIONS
%%%

\section{\AUCTeX{} Customisations}
\label{sec:alb-corp-layout-documentation:auctex-cust}

Under \AUCTeX{} the file \texttt{alb-corp.el} is automatically loaded
whenever the \texttt{alb-corp} class is used.  The customisation adds
the symbols and environments to \AUCTeX{}.  This provides the following
simple prompting.
\begin{itemize}
\item
  \begin{flushleft}
    Tab completion and argument prompting for the commands:\\
    \texttt{albTitle}, \texttt{albTitlePageSection}, \texttt{albLtxCmd},
    \texttt{albLtxEnv}, \texttt{albLtxOpt}, \texttt{albLtxArg},
    \texttt{albLtxPrm}, \texttt{PDFLaTeX}, \texttt{AUCTeX},
    \texttt{RefTeX}, and \texttt{albLogo}.
  \end{flushleft}

\item
  \begin{flushleft}
    Tab completion for the environments:\\
    \texttt{albTitlePage}.
  \end{flushleft}
\end{itemize}
The customisation is not context sensitive, and will not ensure that
these commands and environments are used correctly.



\end{document}



%%% Local Variables:
%%% mode: latex
%%% TeX-master: t
%%% ispell-local-dictionary: "british"
%%% End:
