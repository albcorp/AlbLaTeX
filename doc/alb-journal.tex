%%%
%%% AlbLaTeX/doc/alb-journal.tex
%%%
%%%     See copyright notice and license in text.
%%%
%%%   - Documentation for the 'alb-journal' LaTeX document class.
%%%



\documentclass[11pt,a4paper,oneside,titlepage]{alb-corp}



%
% URL Typesetting
%
% See: 'url.sty'.

\usepackage{url}


%
% Author Year Citations
%
% See: 'natbib.dvi'.

\usepackage[sectionbib,round]{natbib}


%
% Sloppy Line Breaks
%
% Turn off careful line breaks and hyphenation.

\sloppy



\begin{document}


%%%
%%% TITLE
%%%

\albTitle{%
  Page Layout for Research Journals%
}


\begin{albTitlePage}

  \albTitlePageSection{Author}

  Andrew Lincoln Burrow\\
  \url{albcorp@gmail.com}


  \albTitlePageSection{Abstract}

  The \texttt{AlbJournalLayout} package provides a single
  \texttt{alb-journal} \LaTeX{} document class.  The class implements a
  book design for research journals that is designed to allow the
  collection of citations and notes, where each journal entry is
  recorded under a date.  It sets up the page layout and section
  formatting.  It also provides a single additional \LaTeX{} environment
  to markup a research note.


  \albTitlePageSection{Copyright}

  Copyright \copyright{} 2005--2008, 2013 Andrew Lincoln Burrow.\\
  This program may be distributed and/or modified under the conditions
  of the \LaTeX{} Project Public License, either version 1.3 of this
  license or (at your option) any later version.

  \medskip{}

  The latest version of this license is in
  \begin{quote}
    \url{http://www.latex-project.org/lppl.txt}
  \end{quote}
  and version 1.3 or later is part of all distributions of LaTeX version
  2005/12/01 or later.

  \medskip{}

  This work has the LPPL maintenance status `author-maintained'.

  \medskip{}

  This work consists of the files
  \begin{quote}
    \begin{flushleft}
      \url{alb-algorithms.sty}, \url{alb-avm.sty}, \url{alb-corp.cls},
      \url{alb-float-tools.sty}, \url{alb-graph-theory.sty},
      \url{alb-journal.cls}, \url{alb-order-theory.sty},
      \url{alb-proofs.sty}, \url{alb-theorems.sty},
      \url{alb-thesis.cls}, \url{alb-algorithms.tex}, \url{alb-avm.tex},
      \url{alb-corp.tex}, \url{alb-float-tools.tex},
      \url{alb-graph-theory.tex}, \url{alb-journal.tex},
      \url{alb-order-theory.tex}, \url{alb-proofs.tex},
      \url{alb-theorems.tex}, \url{alb-thesis.tex}.
      \url{alb-journal-glossary.ist}, \url{alb-journal-index.ist},
      \url{alb-thesis-glossary.ist}, and \url{alb-thesis-index.ist}.
    \end{flushleft}
  \end{quote}


  \albTitlePageSection{Version Information}

  \verb$Revision$\\
  \verb$Date$

\end{albTitlePage}



%%%
%%% INTRODUCTION
%%%

\section{Introduction}
\label{sec:alb-journal:intr}

The \texttt{AlbJournalLayout} package is designed to assist in the
keeping of research journals.  It consists of a single document class
\texttt{alb-journal}.  In particular, it designed to achieve consistency
and to provide guidance when writing research journals.

The document class sets the document appearance, so that sectioning
commands and citations produce a well designed reference that is easy to
browse.  This is achieved through the \texttt{natbib} and
\texttt{chapterbib} \LaTeX{} packages, and by assuming that the chapter
titles are dates.  The page layout provides a wide external margin.
Into this margin is set the current chapter title.  Hence, if each
chapter title is a date of the form \emph{01 MAR 2005}, it is a simple
matter to browse the dates in the bound document.  The page number is
placed directly under the outer edge of the text, and appears on all
pages including the first page in each chapter.  Finally, it is easy to
arrange the bibliography at the end of each chapter with the \LaTeX{}
package \texttt{chapterbib}, to produce an index of citations by name at
the end of the document with the \LaTeX{} package \texttt{natbib}, and
to use the glossary command to add indexed keywords to each research
note.



%%%
%%% SUGGESTED DOCUMENT STRUCTURE
%%%

\section{Suggested Document Structure}
\label{sec:alb-journal:sugg-docum-struct}

To enjoy the benefits of the \texttt{alb-journal} document class, the
document structure must conform to certain rules.  Additional steps,
suggested here, will make the system easy to use.  These guidelines are
presented here in terms of page layout options, preamble commands, and
document structure.



\subsection{Page Layout Options}
\label{sec:alb-journal:page-layout-opti}

The \texttt{alb-journal} document class attempts to respect page layout
options.  In particular, you can use \texttt{oneside}, and
\texttt{twoside}.  It makes little sense to declare a large document
\texttt{notitlepage}.  Furthermore, the page layout will not accomodate
\texttt{twocolumn} or \texttt{reversemp}.  The page layout accomodates
\texttt{a4paper} and \texttt{letterpaper} page sizes.

Page layout options must be placed in the \texttt{documentclass}
declaration, and each relevant option should be explicitly declared
since the \texttt{AlbLaTeXDocumentTemplate} makefile parses document
class options.  The following content is a typical example of the
document class declaration.
\begin{quote}
\begin{verbatim}
\documentclass[11pt,a4paper,oneside,titlepage]{alb-corp}
\end{verbatim}
\end{quote}



\subsection{Indexing Commands}
\label{sec:alb-journal:index-comm}

The excellent \texttt{natbib} and \texttt{chapterbib} \LaTeX{} packages
are recommended, along with the setup of the keyword and citation
indices.

The packages are included by the following preamble commands.
\begin{quote}
\begin{verbatim}
\usepackage[sectionbib]{natbib}
\usepackage{chapterbib}
\usepackage{url}
\end{verbatim}
\end{quote}
The \texttt{sectionbib} option ensures that the bibliography is placed
under a section rather than a chapter, so that each entry should itself
be organised into a chapter.  The \texttt{url} package improves the
typesetting of URLs, which are increasingly common in bibliographic
material.

The keyword and citation indices are setup by the following preamble
commands.
\begin{quote}
\begin{verbatim}
\citeindextrue
\makeglossary
\makeindex
\end{verbatim}
\end{quote}
The \albLtxCmd{citeindextrue} command causes each citation to generate
an index entry and the author of the cited work.



\subsection{Document Structure}
\label{sec:alb-journal:docum-struct}

In order to generate a bibliography for each journal entry, it is
necessary to divide each journal entry into its own file and chapter.
The master document includes these entry files, along with the files
defining the title page, table of contents, and indices.  It is best to
make chapter titles and filenames agree, and to base these names on the
date of the journal entry.

The following examples shows such a document structure.  The master file
contains the preamble and the following document environment, where the
list of included files in the main matter section varies.
\begin{quote}
\begin{verbatim}
\begin{document}
  \frontmatter
  \include{title_page}
  \include{table_cont}

  \mainmatter
  \include{read_01_mar_2005}
  \include{read_03_apr_2005}

  \backmatter
  \include{indic}
\end{document}
\end{verbatim}
\end{quote}
The title page is up to the author, but the contents page is a very
simple file containing only the following command.
\begin{quote}
\begin{verbatim}
\tableofcontents
\end{verbatim}
\end{quote}
Likewise, the index is a very simple file containing only the following
commands.
\begin{quote}
\begin{verbatim}
\chapter{Keyword Index}
\printglossary

\chapter{Citation Index}
\printindex
\end{verbatim}
\end{quote}

The document structure is completed by the inclusion of journal entries.
Each journal entry is a file that must be included by the master
document.  It begins with a chapter heading indicating the date, and
concludes with commands to generate a biblography in the chapter.
Research notes are marked up via the \texttt{albResearchNote}
environment described in
Section~\ref{sec:alb-journal:using-comm-envir}.

The following content is a minimal example of a chapter.  The use of
\texttt{chapter} to markup the date is mandatory, but the choice of
sectioning to set out the themes is up to the author.  Finally, the
bibliography commands are required in each chapter, and the style must
be consistent throughout the document.
\begin{quote}
\begin{verbatim}
\chapter{01 MAR 2005}

Papers collected in the preparation of the grant application.

\section{Wiki Research at SIAL}

\begin{albResearchNote}%
    {\citet{burrow04:_negot_acces_wiki}}%
    {\glossary{wiki}, \glossary{access~rules}}
\item Describes the negotiation of access rules in wiki
\end{albResearchNote}

\bibliographystyle{plainnat}
\bibliography{alb-bibliography}
\end{verbatim}
\end{quote}



%%%
%%% USING THE COMMANDS AND ENVIRONMENTS
%%%

\section{Using the Commands and Environments}
\label{sec:alb-journal:using-comm-envir}


The single \texttt{albResearchNote} list environment of the
\texttt{alb-journal} concerns the markup of notes for a source.

\begin{description}
\item[\albLtxEnv{albResearchNote}\albLtxArg{source}\albLtxArg{keywords}]
  Typeset a research note identified by a citation and tagged with a
  collection of keywords.  The first argument should contain a
  \albLtxCmd{citet} command.  The second argument should contain
  \albLtxCmd{glossary} commands.  The environment generates typeset
  output for \albLtxCmd{glossary} commands in the second argument, but
  does not parse structure in the entry.  Therefore, use only simple
  glossary entries in this context, and write phrases using \verb|~| as
  a separator character.

  An important note from \citet[p99]{booth03:_craft_resear}
  \begin{quote}
    ``Clearly and unambiguously distinguish four kinds of references:
    what you quote directly, what you paraphrase, what you summarize,
    and what you write as your own thoughts.''
  \end{quote}
\end{description}



%%%
%%% MAKEFILE TARGETS
%%%

\section{Makefile Targets}
\label{sec:alb-journal:makef-targ}

Given documents of the \texttt{alb-journal} class are expected to
contain an index, it makes good sense to use the makefile from the
\texttt{AlbLaTeXDocumentTemplate} makefile.  This makefile will detect
the existence of index commands and take the appropriate steps.  The
target \texttt{all} is a synonym for \texttt{idx}, which generates the
final PDF document such that the index references are correct.



%%%
%%% BIBLIOGRAPHY
%%%

\bibliographystyle{plainnat}
\bibliography{alb-bibliography}



\end{document}



%%% Local Variables:
%%% mode: latex
%%% TeX-master: t
%%% ispell-local-dictionary: "british"
%%% End:
