%%%
%%% alb-latex/doc/user/alb-thesis.tex
%%%
%%%     See copyright notice and license in text.
%%%
%%%   - Documentation for the 'alb-thesis' LaTeX document class.
%%%



\documentclass[11pt,a4paper,oneside,titlepage]{alb-latex}



%
% URL Typesetting
%
% See: 'url.sty'.

\usepackage{url}


%
% Sloppy Line Breaks
%
% Turn off careful line breaks and hyphenation.

\sloppy



\begin{document}



%%%
%%% TITLE
%%%

\albTitle{%
  Page Layout for a PhD Thesis%
}

\begin{albTitlePage}

  \albTitlePageSection{Author}

  Andrew Lincoln Burrow

  \albTitlePageSection{Abstract}

  The \texttt{AlbThesisLayout} package provides a single
  \texttt{alb-thesis} \LaTeX{} document class.  The class implements a
  book design for scholarly work in mathematical and computer science.
  It sets up the page layout and section formatting.  It does not
  provide additional \LaTeX{} commands or environments.


  \albTitlePageSection{Copyright}

  Copyright \copyright{} 2003--2006, 2013, 2017 Andrew Lincoln Burrow.\\
  This program may be distributed and/or modified under the conditions
  of the \LaTeX{} Project Public License, either version 1.3 of this
  license or (at your option) any later version.

  \medskip{}

  The latest version of this license is in
  \begin{quote}
    \url{http://www.latex-project.org/lppl.txt}
  \end{quote}
  and version 1.3 or later is part of all distributions of LaTeX version
  2005/12/01 or later.

  \medskip{}

  This work has the LPPL maintenance status `author-maintained'.

  \medskip{}

  This work consists of the files
  \begin{quote}
    \begin{flushleft}
      \url{alb-algorithms.sty}, \url{alb-avm.sty}, \url{alb-latex.cls},
      \url{alb-float-tools.sty}, \url{alb-graph-theory.sty},
      \url{alb-journal.cls}, \url{alb-order-theory.sty},
      \url{alb-proofs.sty}, \url{alb-theorems.sty},
      \url{alb-thesis.cls}, \url{alb-algorithms.tex}, \url{alb-avm.tex},
      \url{alb-latex.tex}, \url{alb-float-tools.tex},
      \url{alb-graph-theory.tex}, \url{alb-journal.tex},
      \url{alb-order-theory.tex}, \url{alb-proofs.tex},
      \url{alb-theorems.tex}, \url{alb-thesis.tex}.
      \url{alb-journal-glossary.ist}, \url{alb-journal-index.ist},
      \url{alb-thesis-glossary.ist}, and \url{alb-thesis-index.ist}.
    \end{flushleft}
  \end{quote}


  \albTitlePageSection{Version Information}

  \verb$Revision$\\
  \verb$Date$

\end{albTitlePage}



%%%
%%% INTRODUCTION
%%%

\section{Introduction}
\label{sec:alb-thesis:intr}

The \texttt{AlbThesisLayout} package implements a book design suitable
for a thesis or other large scholarly document.  It consists of a single
document class \texttt{alb-thesis}.  It does not provide general purpose
commands or environments.

The page layout provides an allocation for margin notes on the outer
edge.  Several elements exploit this space.  Chapter headings are right
aligned against the outer edge, so that they access this space.  Large
floats access the space via the \texttt{albInflate} environment of the
\texttt{AlbFloatTools} package.  The page numbers of referenced
propositions are displayed in the space, if the reference is one of the
prefixed reference commands such as \albLtxCmd{albDRef} from the
\texttt{AlbTheorems} package.

Page numbers are placed in the running headers with the chapter and
section information.  In twoside mode, the left header contains the page
number, chapter title, and chapter number, while the right header
contains the section number, section title, and page number.  In each
case, the space for margin notes accomodates the page number.



%%%
%%% SUGGESTED DOCUMENT STRUCTURE
%%%

\section{Suggested Document Structure}
\label{sec:alb-thesis:sugg-docum-struct}

The following guidelines reflect the design goals of the
\texttt{alb-thesis} document class.  They are presented in terms of page
layout options.  Although, all large documents should exploit the
\albLtxCmd{include} command, this aspect of document structure is not
discussed here.



\subsection{Page Layout Options}
\label{sec:alb-thesis:page-layout-opti}

The \texttt{alb-thesis} document class attempts to respect page layout
options.  In particular, you can use of \texttt{oneside}, and
\texttt{twoside} as global options.  It makes little sense to declare a
large document \texttt{notitlepage}.  Furthermore, the page layout will
not accomodate \texttt{twocolumn} or \texttt{reversemp}.  The page
layout accomodates \texttt{a4paper} and \texttt{letterpaper} page sizes.

Page layout options must be placed in the \texttt{documentclass}
declaration, and each relevant option should be explicitly declared
since the \texttt{AlbLaTeXDocumentTemplate} makefile parses document
class options.  The following content is a typical example of the
document class declaration.
\begin{quote}
\begin{verbatim}
\documentclass[10pt,a4paper,twoside]{alb-thesis}
\end{verbatim}
\end{quote}



%%%
%%% MAKEFILE TARGETS
%%%

\section{Makefile Targets}
\label{sec:alb-thesis:makef-targ}

Given documents of the \texttt{alb-thesis} class are expected to contain
an index, it makes good sense to use the makefile from the
\texttt{AlbLaTeXDocumentTemplate} package.  This makefile will detect
the existence of index commands and take the appropriate steps.  The
target \texttt{all} is a synonym for \texttt{idx}, which generates the
final PDF document such that the index references are correct.



\end{document}



%%% Local Variables:
%%% mode: latex
%%% TeX-master: t
%%% ispell-local-dictionary: "british"
%%% End:
